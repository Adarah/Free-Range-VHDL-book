% Free range VHDL
% Authors: Bryan Mealy, Fabrizio Tappero
% Date: May, 2012
% URL: freerangefactory.org
% (C) 2013 B. Mealy, F. Tappero
%
% !TEX root = master.tex
%
\chapter{VHDL Design Units}
The ``black-box'' approach to any type of design implies a hierarchical structure in which varying amounts of detail are available at each of the different levels of the hierarchy. In the black-box approach, units of action which share a similar purpose are grouped together and abstracted to a higher level. Once this is done, the module is referred to by its inherently more simple black-box representation rather than by the details of the circuitry that actually performs that functionality. This approach has two main advantages. First, it simplifies the design from a systems standpoint. Examining a circuit diagram containing appropriately named black boxes is much more understandable than staring at a circuit containing a countless number of logic gates. Second, the black-box approach allows for the reuse of previously written code.

Not surprisingly, VHDL descriptions of circuits are based on the black-box approach. The two main parts of any hierarchical design are the black box and the stuff that goes in the black box (which can of course be other black boxes). In VHDL, the black box is referred to as \textbf{entity} and the stuff that goes inside it is referred to as the \textbf{architecture}. For this reason, the VHDL entity and architecture are closely related. As you can probably imagine, creating the entity is relatively simple while a good portion of the VHDL coding time is spent on properly writing the architecture. Our approach here is to present an introduction to writing VHDL code by describing the entity and then moving on to the details of writing the architecture. Familiarity with the entity will hopefully aid in your learning of the techniques to describe the architecture.

\section{Entity}
The VHDL entity construct provides a method to abstract the functionality of a circuit description to a higher level. It provides a simple wrapper for the lower-level circuitry. This wrapper effectively describes how the black box interfaces with the outside world. Since VHDL describes digital circuits, the entity simply lists the various inputs and outputs of the underlying circuitry. In VHDL terms, the black box is described by an entity declaration. The syntax of the entity declaration is shown in Listing~\ref{entity_basic}.

\noindent
\begin{minipage}{0.99\linewidth}
\begin{lstlisting}[label=entity_basic, caption=The entity declaration in VHDL., mathescape=true]
entity my_entity is
port(
     port_name_1 : in    std_logic ;
     port_name_2 : out   std_logic;
     port_name_3 : inout std_logic ); --do not forget the semicolon
end my_entity; -- do not forget this semicolon either
\end{lstlisting}
\end{minipage}

\texttt{my\_entity} defines the name of the entity. The next section is nothing more than the list of signals from the underlying circuit that are available to the outside world, which is why it is often referred to as an interface specification. The \texttt{port\_name\_x} is an identifier used to differentiate the various signals. The next keyword (the keyword \texttt{in}) specifies the direction of the signal relative to the entity where signals can either enter, exit or do both. These input and output signals are associated with the keywords \textbf{in}, \textbf{out} and \textbf{inout}\footnote{The \texttt{inout} data mode will be discussed later on in the book.} respectively. The next keyword (the keyword \texttt{std\_logic}) refers to the type of data that the port will handle. There are several data types available in VHDL but we will primarily deal with the \texttt{std\_logic} type and derived versions. More information regarding the various VHDL data types will be discussed later.

When you attempt to write fairly complicated VHDL code, you will need to split your code into different files, functions and packages constructors which will help you better deal with your code. In this scenario, the entity body will not only host the port definition statements but, most likely, other procedures as well. We will talk about this later in the book.

\noindent
\begin{minipage}{0.5\linewidth}
	\vspace{5pt}
	\begin{lstlisting}[label=entity_2, caption=VHDL entity declaration.]
------------------------------
--  interface description   --
--  of killer_ckt           --
------------------------------
entity killer_ckt is
port (
life_in1       : in  std_logic;
life_in2       : in  std_logic;
crtl_a, ctrl_b : in  std_logic;
kill_a         : out std_logic;
kill_b, kill_c : out std_logic);
end killer_ckt;
	\end{lstlisting}
\end{minipage}
\begin{minipage}{0.49\linewidth}
	\begin{flushright}
		\begin{tikzpicture}[x=1mm,y=1mm,line width=0.8pt,scale=0.9,framed]
			%\draw[help lines] (0,0) grid (50,50);
			% BOX
			\draw (20,0) rectangle (37,35) node[midway]{killer\_ckt};
			% INPUTS
			\small
			\node (a) at (20,-2.5) {}; % this is the reference point
			\draw [latex-] ($(a)+(0,25)$) -- ++(-10,0) node[left]{life\_in1};
			\draw [latex-] ($(a)+(0,20)$) -- ++(-10,0) node[left]{life\_in2};
			\draw [latex-] ($(a)+(0,15)$) -- ++(-10,0) node[left]{ctrl\_a};
			\draw [latex-] ($(a)+(0,10)$) -- ++(-10,0) node[left]{ctrl\_b};
			% OUTPUTS
			\draw [-latex] ($(a)+(17,25)$) -- ++(10,0) node[right]{kill\_a};
			\draw [-latex] ($(a)+(17,20)$) -- ++(10,0) node[right]{kill\_b};
			\draw [-latex] ($(a)+(17,15)$) -- ++(10,0) node[right]{kill\_c};
		\end{tikzpicture}
	\end{flushright}
\end{minipage}

Listing~\ref{entity_2} shows an example of a black box and the VHDL code used to describe it. Listed below are a few points to note about the code in Listing~\ref{entity_2}. Most of the points deal with the readability and understandability of the VHDL code.

\begin{my_list}
\item Each port name is unique and has an associated mode and data type. This is a requirement.
\item The VHDL compiler allows several port names to be included on a single line. Port names are separated by commas. Always strive for readability.
\item Port names are somewhat lined up in a feeble attempt to increase readability. This is not a requirement but you should always be striving for readability. Remember that white spaces are ignored by the compiler.
\item A comment, which tells us what this entity does, is included.
\item A black-box diagram of the circuit is also provided. Once again, drawing some type of diagram helps with any VHDL code that you may be writing. Remember: do not be scared, draw a picture.
\end{my_list}

Hopefully, you are not finding these entity specifications too challenging. In fact, they are so straightforward, we will throw in one last twist before we leave the realm of VHDL entities. Most of the more meaningful circuits that you will be designing, analyzing and testing have many similar and closely related inputs and outputs. These are commonly referred to as ``bus signals'' in computer lingo. Bus lines are made of more than one signal that differ in name by only a numeric character. In other words, each separate signal in the bus name contains the bus name plus a number to separate it from other signals in the bus. Individual bus signals are referred to as elements of the bus. As you would imagine, buses are often used in digital circuits. Unfortunately, the word bus also refers to established data transfer protocols. To disambiguate the word bus, we will be using the word ``bundle'' to refer to a set of similar signals and bus to refer to a protocol.

Bundles are easily described in the VHDL entity. All that is needed is a new data type and a special notation to indicate when a signal is a bundle or not. A few examples are shown in Listing~\ref{entity_3}. In these examples note that the mode remains the same but the type has changed. The \texttt{std\_logic} data type has now been replaced by the word \texttt{std\_logic\_vector} to indicate that each signal name contains more than one signal. There are ways to reference individual members of each bundle, but we will get to those details later.

As you can see by examining Listing~\ref{entity_3}, there are two possible methods to describe the signals in a bundle. These two methods are shown in the argument lists that follow the data type declaration. The signals in the bundle can be listed in one of two orders which are specified by the \textbf{to} and \textbf{downto} keywords. If you want the most significant bit of your bundle to be on the the first bit on the left you use \textbf{downto} keyword. Be sure not to forget the orientation of signals when you are using this notation in your VHDL model.

In the black box of Listing~\ref{entity_3} you can see the formal notation for a bundle. Note that the black box uses a slash-and-number notation. The slash across the signal line indicates the signal is a bundle and the associated number specifies the number of signals in the bundle. Worthy of mention regarding the black box relative to Listing~\ref{entity_3} is that the input lines \texttt{sel1} and \texttt{sel0} could have been made into one bundle containing the two signals.

\noindent
\begin{minipage}{0.99\linewidth}
\vspace{5pt}
\centering
\begin{tikzpicture}[x=1mm,y=1mm,line width=0.8pt,scale=0.9,framed]
%\draw[help lines] (0,0) grid (50,50);
% BOX
\draw (20,0) rectangle (35,35) node[midway]{mux\_4};
% INPUTS
\small
\node (a) at (20,-2.5) {}; % this is the reference point
\draw [latex-] ($(a)+(0,35)$) -- ++(-10,0) node[left]{a\_data} node[pos=0.4,above]{8} node[pos=0.6]{/};
\draw [latex-] ($(a)+(0,30)$) -- ++(-10,0) node[left]{b\_data} node[pos=0.4,above]{8} node[pos=0.6]{/};
\draw [latex-] ($(a)+(0,25)$) -- ++(-10,0) node[left]{c\_data} node[pos=0.4,above]{8} node[pos=0.6]{/};
\draw [latex-] ($(a)+(0,20)$) -- ++(-10,0) node[left]{d\_data} node[pos=0.4,above]{8} node[pos=0.6]{/};
%\draw [latex-] ($(a)+(0,15)$) -- ++(-10,0) node[left]{sel1};
\draw [latex-] ($(a)+(0,10)$) -- ++(-10,0) node[left]{sel0};
\draw [latex-] ($(a)+(0,5)$) -- ++(-10,0) node[left]{sel1};
% OUTPUTS
%\draw [-latex] ($(a)+(15,30)$) -- ++(10,0) node[right]{data\_out1} node[pos=0.6,above]{8} node[pos=0.4]{/};
%\draw [-latex] ($(a)+(15,25)$) -- ++(10,0) node[right]{data\_out2} node[pos=0.6,above]{8} node[pos=0.4]{/};
\draw [-latex] ($(a)+(15,20)$) -- ++(10,0) node[right]{data\_out} node[pos=0.6,above]{8} node[pos=0.4]{/};
%\draw [-latex] ($(a)+(15,15)$) -- ++(10,0) node[right]{data\_out4} node[pos=0.6,above]{8} node[pos=0.4]{/};
%\draw [-latex] ($(a)+(15,10)$) -- ++(10,0) node[right]{data\_out5} node[pos=0.6,above]{8} node[pos=0.4]{/};
\end{tikzpicture}
\begin{lstlisting}[label=entity_3, caption=Entity declaration with bundles.]
-------------------------------------------------------------
-- Unlike the other examples, this is actually an interface
-- for a MUX that selects one of four bus lines for the output.
-------------------------------------------------------------
entity mux4 is
port (  a_data    : in     std_logic_vector(0 to 7);
        b_data    : in     std_logic_vector(0 to 7);
        c_data    : in     std_logic_vector(0 to 7);
        d_data    : in     std_logic_vector(0 to 7);
        sel1,sel0 : in     std_logic;
        data_out  : out    std_logic_vector(7 downto 0));
end mux4;
\end{lstlisting}
\end{minipage}

The data type \texttt{std\_logic} and the data type \texttt{std\_logic\_vector} is what the IEEE has standardized for the representation of digital signals. Normally, you should consider that these data types assume the logic value \texttt{1} or the logic value \texttt{0}. However, as specified in the \texttt{std\_logic\_1164} package, the implementation of the \texttt{std\_logic} type (and the \texttt{std\_logic\_vector} type) is a little more generous and includes 9 different values, specifically: \texttt{0,1,U,X,Z,W,L,H,-}.

The data type \texttt{std\_logic} becomes available to you soon after the declaration \texttt{library IEEE; use IEEE.std\_logic\_1164.all;} at the beginning of your code.

The reason for all these values is the desire for modeling three-state drivers, pull-up and pull-down outputs, high impedance state and a few others types of inputs/outputs. For more details refer to the IEEE 1164 Standard\footnote{IEEE 1164 Standard \footnotesize\url{http://en.wikipedia.org/wiki/IEEE_1164}}.

Alternatively to the \texttt{std\_logic} data type, VHDL programmers sometimes use the much simpler data type \texttt{bit} which has only the logic values \texttt{1} and \texttt{0}.

\section{VHDL Standard Libraries}
The VHDL language as many other computer languages, has gone through a long and intense evolution. Among the most important standardization steps we can mention are the release of the \texttt{IEEE Standard 1164} package as well as some child standards that further extended the functionality of the language. In order to take advantage of the main implementable feature of VHDL you just need to import the two main library packages as shown in lines 2$\sim$4 of Listing~\ref{main_vhdl_lib}.

\noindent
\begin{minipage}{0.99\linewidth}
\begin{lstlisting}[numbers=left,label=main_vhdl_lib, caption=Typical inclusions of IEEE standard libraries.]
-- library declaration
library IEEE;
use IEEE.std_logic_1164.all; -- basic IEEE library
use IEEE.numeric_std.all;    -- IEEE library for the unsigned type and
                             -- various arithmetic operators

-- WARNING: in general try NOT to use the following libraries
--          because they are not IEEE standard libraries
-- use IEEE.std_logic_arith.all;
-- use IEEE.std_logic_unsigned.all;
-- use IEEE.std_logic_signed

-- entity
entity my_ent is
	port ( A,B,C : in  std_logic;
	       F     : out std_logic);
end my_ent;
-- architecture
architecture my_arch of my_ent is
    signal v1,v2 : std_logic_vector (3 downto 0);
    signal u1    : unsigned (3 downto 0);
    signal i1    : integer;
begin
    u1 <= "1101";
    i1 <= 13;
    v1 <= std_logic_vector(u1);                         -- = "1101"
    v2 <= std_logic_vector(to_unsigned(i1, v2'length)); -- = "1101"

-- "4" could be used instead of "v2'length", but the "length"
-- attribute makes life easier if you want to change the size of v2

	F <= NOT (A AND B AND C);
end my_arch;
\end{lstlisting}
\end{minipage}

Once these packages have been included, you will have access to a very large set of goodies: several data types, overloaded operators, various conversion functions, math functions and so on. For instance, the inclusion of the package \texttt{numeric\_std.all} will give you the possibility of using the \texttt{unsigned} data type and the function \texttt{to\_unsigned} shown in Listing~\ref{main_vhdl_lib}. For a detailed description of what these libraries include, refer to the \texttt{Language Templates} of your favorite synthesis software tool.

For more information on VHDL standard libraries refer to the Appendix.

\section{Architecture}
The VHDL entity declaration, introduced before, describes the interface or the external representation of the circuit. The architecture describes what the circuit actually does. In other words, the VHDL architecture describes the internal implementation of the associated entity. As you can probably imagine, describing the external interface to a circuit is generally much easier than describing how the circuit is intended to operate. This statement becomes even more important as the circuits you are describing become more complex.

There can be any number of equivalent architectures describing a single entity. As you will eventually discover, the VHDL coding style used inside the architecture body has a significant effect on the way the circuit is synthesized (how the circuit will be implemented inside an actual silicon device). This gives the VHDL programmer the flexibility of designing systems with specific positive or negative features such as particular physical size (measuring the number of needed basic digital elements) or operational speed.

For various reasons, such as facilitating code re-usability and connectibility, an architecture can be modeled in different ways. Understanding the various modeling techniques and understanding how to use them represent the first important steps in learning VHDL.

An architecture can be written by means of three modeling techniques plus any combination of these three. There is the \textbf{data-flow model}, the \textbf{behavioral model}, the \textbf{structural model} and the \textbf{hybrid models}. These models will be described throughout the book. Listing~\ref{architecture_1} gives a sneak preview of what a simple but complete VHDL code block looks like.

\begin{lstlisting}[float, numbers=left, label=architecture_1, caption=Example of a simple VHDL block.]
---------------------- FILE: my_sys.vhd ----------------------
-- library declaration
library ieee;
use ieee.std_logic_1164.all;

-- the ENTITY
entity circuit1 is
port (
    A,B,C   : in  std_logic;
    F, G    : out std_logic);
end circuit1;

-- the ARCHITECTURE
architecture circuit1_arc of circuit1 is
	signal sig_1 : std_logic;             -- signal definition
begin
    process (a,b,c)
        variable var_1 : integer;         -- variable definition
    begin
        F <= not (A and B and C);         -- signal assignment
	    sig_1 <= A;                       -- another signal assignment
	    var_1 := 34;                      -- variable assignment
    end process;
    G <= not (A and B);                   -- concurrent assignment
end circuit1_arc;
\end{lstlisting}

\section{Signal and Variable Assignments}
In VHDL there are several object types. Among the most frequently used we will mention the \textbf{signal} object type, the \textbf{variable} object type and the \textbf{constant} object type. The signal type is the software representation of a wire. The variable type, like in C or Java, is used to store local information. The constant is like a variable object type, the value of which cannot be changed. A signal object can be of different types; we saw before, for example, that a signal object can be of type \texttt{std\_logic} or of other types like integer, custom types, etc. The same applies for variable objects.

Before using any signal or variable, it is mandatory to declare them. Signals are declared at the top of the architecture body, just before the keyword \texttt{begin}. Variables must be declared inside the \textit{process construct} and are local. An example is shown in line 15 and 18 of Listing~\ref{architecture_1}.

As seen in line 20 and line 21 of Listing~\ref{architecture_1} when you want to assign a new value to an object of type signal you use the operator ``$<=$''. Alternatively, when you want to assign a new value to an object of type variable you will use the operator ``$:=$'', shown in line 22.

It is important to understand the difference between variables and signals, specifically when their value changes. A variable changes its value soon after the variable assignment is executed. Instead, a signal changes its value ``some time'' after the signal assignment expression is evaluated. This has important consequences for the updated values of variables and signals. This means that you should never assume that a signal assignment can happen instantly and it also means that you can take advantage of variables every time you need to implement a counter or to store values when inside a process.

In order to be able to introduce the use of a variable we had to employ the \textit{process construct}, a construct that you are not yet familiar with. We will see more in details later on in the book that any time we need a non-concurrent execution environment where code lines are executed one after the other (like in C or Java), we will be using the \textit{process construct}. Inside a process, all instructions are executed consecutively from top to bottom. However the process itself will be executed concurrently with the rest of the code (e.g. the instruction at line 24).

Always remember that the assignment of Listing~\ref{architecture_1} at line 24 and the execution of the process, are not executed consecutively but instead concurrently (all at the same time). Any hope that the execution of line 24 will happen before or after the execution of the process will only result in great disappointment.

As a final note, let us to remind that the type \texttt{std\_logic} only exists if you declare the library \texttt{ieee.std\_logic\_1164.all} as done in line 4 of Listing~\ref{architecture_1}.

\section{Summary}
\begin{my_list}
\item The entity declaration describes the inputs and outputs of your circuit. This set of signals is often referred to as the interface to your circuit since these signals are what the circuitry, external to the entity, uses to interact with your circuit.
\item Signals described in the entity declaration include a \textit{mode} specifier and a \textit{type}. The mode specifier can be either an \textit{in} or an \textit{out} (or, as we will see later on, even an \textit{inout}) while the type is either a \texttt{std\_logic} or \texttt{std\_logic\_vector}.
\item The word bundle is preferred over the word bus when dealing with multiple signals that share a similar purpose. The word bus has other connotations that are not consistent with the bundle definition.
\item Multiple signals that share a similar purpose should be declared as a bundle using a \texttt{std\_logic\_vector} type. Bundled signals such as these are always easier to work with in VHDL compared to scalar types such as \texttt{std\_logic}.
\item The architecture describes what your circuit actually does and what its behavior is. Several possible implementations (models) of the same behavior are possible in VHDL. These are the \textbf{data-flow} model, the \textbf{behavioral} model, the \textbf{structural} model as well as any combination of them, generally called \textbf{hybrid} model.
\end{my_list}

\section{Exercises}
\begin{my_num_list}
\item What is referred to by the word bundle?
\item What is a common method of representing bundles in black-box diagrams?
\item Why is it considered a good approach to always draw a black-box diagram when using VHDL to model digital circuits?
\item Write VHDL entity declarations that describe the following black-box diagrams:

\noindent
\begin{minipage}{0.5\linewidth}
	\vspace{5pt}
	a)\\
	\begin{tikzpicture}[x=1mm,y=1mm,line width=0.8pt,scale=0.8,framed]
		%\draw[help lines] (0,0) grid (50,50);
		% BOX
		\draw (20,5) rectangle (37,25) node[midway]{sys1};
		% INPUTS
		\small
		\node (hide) at (0,30) {}; % just to expand background
		\node (a) at (20,-2.5) {}; % this is the reference point
		\draw [latex-] ($(a)+(0,25)$) -- ++(-10,0) node[left]{a\_in1};
		\draw [latex-] ($(a)+(0,20)$) -- ++(-10,0) node[left]{b\_in2};
		\draw [latex-] ($(a)+(0,15)$) -- ++(-10,0) node[left]{clk};
		\draw [latex-] ($(a)+(0,10)$) -- ++(-10,0) node[left]{ctrl\_int};
		% OUTPUTS
		\draw [-latex] ($(a)+(17,20)$) -- ++(10,0) node[right]{out\_b};
	\end{tikzpicture}
\end{minipage}
\begin{minipage}{0.49\linewidth}
	\vspace{5pt}
	b)\\
	\begin{tikzpicture}[x=1mm,y=1mm,line width=0.8pt,scale=0.8,framed]
		%\draw[help lines] (0,0) grid (50,50);
		% BOX
		\draw (20,0) rectangle (37,25) node[midway]{sys2};
		% INPUTS
		\small
		\node (hide) at (0,30) {}; % just to expand background
		\node (a) at (20,-2.5) {}; % this is the reference point
		\draw [latex-] ($(a)+(0,24)$) -- ++(-10,0) node[left]{input\_w};
		\draw [latex-] ($(a)+(0,18)$) -- ++(-10,0) node[left]{a\_data} node[pos=0.4,above]{8} node[pos=0.7]{/};
		\draw [latex-] ($(a)+(0,11)$) -- ++(-10,0) node[left]{b\_data} node[pos=0.4,above]{8} node[pos=0.7]{/};
		\draw [latex-] ($(a)+(0, 5)$) -- ++(-10,0) node[left]{clk};
		% OUTPUTS
		\draw [-latex] ($(a)+(17,19)$) -- ++(10,0) node[right]{dat\_4} node[pos=0.6,above]{8} node[pos=0.4]{/};
		\draw [-latex] ($(a)+(17,10)$) -- ++(10,0) node[right]{dat\_5} node[pos=0.6,above]{3} node[pos=0.4]{/};
	\end{tikzpicture}
\end{minipage}

\item Provide black-box diagrams that are defined by the following VHDL entity declarations:

\noindent
\begin{minipage}{1\linewidth}
a)
\begin{lstlisting}[]

entity ckt_c is
port (
    bun_a, bun_b, bun_c  : in    std_logic_vector(7 downto 0);
    lda, ldb, ldc        : in    std_logic;
    reg_a, reg_b, reg_c  : out   std_logic_vector(7 downto 0));
end ckt_c;
\end{lstlisting}
\end{minipage}

\noindent
\begin{minipage}{1\linewidth}
b)
\begin{lstlisting}[]

entity ckt_e is
port (
    RAM_CS, RAM_WE, RAM_OE  : in std_logic;
    SEL_OP1, SEL_OP2        : in std_logic_vector(3 downto 0);
    RAM_DATA_IN             : in std_logic_vector(7 downto 0);
    RAM_ADDR_IN             : in std_logic_vector(9 downto 0);
    RAM_DATA_OUT            : out std_logic_vector(7 downto 0));
end ckt_e;
\end{lstlisting}
\end{minipage}

\item The following two entity declarations contain two of the most common syntax errors made in VHDL. What are they?

\noindent
\begin{minipage}{1\linewidth}
a)
\begin{lstlisting}[framexrightmargin=-180pt]

entity ckt_a is
port (
    J,K : in  std_logic;
    CLK : in  std_logic
    Q   : out std_logic;)
end ckt_a;
\end{lstlisting}
\end{minipage}

\noindent
\begin{minipage}{1\linewidth}
b)
\begin{lstlisting}[framexrightmargin=-80pt, mathescape=true]

entity ckt_b is
port (
mr_fluffy   : in  std_logic_vector(15 downto 0);
mux_ctrl    : in  std_logic_vector(3 downto 0);
byte_out    : out std_logic_vector(3 downto 0);
end ckt_b;
\end{lstlisting}
\end{minipage}
\end{my_num_list}


