\chapter{Exercise Solutions}
\stoptocwriting

\setcounter{section}{3}
\section*{Chapter 3}
\begin{enumerate}
	\item A "bundle" refers to a series of signal which are related to each other. It is often referred to as a "bus", but is referred to by a different name to prevent confusion.
	
	\item A bundle is often shown in black box diagrams as a wire with a slash on it and a number above it indicating the number of signals contained in the bundle.
	
	\item It is considered good practice to draw a black-box diagram of components before implementing them in VHDL as this helps provide a visual representation of each component. In addition, it helps eliminate confusion from having many logic gates inside and allows for the reuse of components elsewhere in your code.
	
	\item The given black-box drawings can be represented with the following VHDL code.
	
	\noindent
	\begin{minipage}{1\linewidth}
		a)
		\begin{lstlisting}[]		
ENTITY sys1 IS
	PORT (
		a_in1		:	IN	STD_LOGIC;
		b_in2		:	IN	STD_LOGIC;
		clk			:	IN	STD_LOGIC;
		ctrl_int	:	IN	STD_LOGIC;
		out_b		:	OUT STD_LOGIC);
END sys1;
		\end{lstlisting}
	\end{minipage}

	\begin{minipage}{1\linewidth}
		b)
		\begin{lstlisting}[]
ENTITY sys2 IS
	PORT (
		input_w	:	IN	STD_LOGIC;
		a_data	:	IN	STD_LOGIC_VECTOR(7 DOWNTO 0);
		b_data	:	IN	STD_LOGIC_VECTOR(7 DOWNTO 0);
		clk		:	IN	STD_LOGIC;
		dat_4	:	OUT	STD_LOGIC_VECTOR(7 DOWNTO 0);
		dat_5	:	OUT	STD_LOGIC_VECTOR(2 DOWNTO 0));
END sys2;
		\end{lstlisting}
	\end{minipage}

	\item The black-box figures are shown below.
	
	\begin{minipage}{1\linewidth}
		\vspace{5pt}
		a)\\
		\begin{tikzpicture}[x=1mm,y=1mm,line width=0.8pt,scale=0.8,framed]
		%\draw[help lines] (0,0) grid (50,50);
		% BOX
		\draw (20,-5) rectangle (37,25) node[midway]{ckt\_c};
		% INPUTS
		\small
		\node (hide) at (0,30) {}; % just to expand background
		\node (a) at (20,-2.5) {}; % this is the reference point
		\draw [latex-] ($(a)+(0,25)$) -- ++(-10,0) node[left]{bun\_a}
		node[pos=0.4,above]{8} node[pos=0.7]{/};
		\draw [latex-] ($(a)+(0,20)$) -- ++(-10,0) node[left]{bun\_b}
		node[pos=0.4,above]{8} node[pos=0.7]{/};
		\draw [latex-] ($(a)+(0,15)$) -- ++(-10,0) node[left]{bun\_c}
		node[pos=0.4,above]{8} node[pos=0.7]{/};
		\draw [latex-] ($(a)+(0,10)$) -- ++(-10,0) node[left]{lda};
		\draw [latex-] ($(a)+(0,5)$) -- ++(-10,0) node[left]{ldb};
		\draw [latex-] ($(a)+(0,0)$) -- ++(-10,0) node[left]{ldc};
		% OUTPUTS
		\draw [-latex] ($(a)+(17,19)$) -- ++(10,0) node[right]{reg\_a} node[pos=0.6,above]{8} node[pos=0.4]{/};
		\draw [-latex] ($(a)+(17,12)$) -- ++(10,0) node[right]{reg\_b} node[pos=0.6,above]{8} node[pos=0.4]{/};
		\draw [-latex] ($(a)+(17,5)$) -- ++(10,0) node[right]{reg\_c} node[pos=0.6,above]{8} node[pos=0.4]{/};
		\end{tikzpicture}
	\end{minipage}

	\begin{minipage}{1\linewidth}
		\vspace{5pt}
		b)\\
		\begin{tikzpicture}[x=1mm,y=1mm,line width=0.8pt,scale=0.8,framed]
		%\draw[help lines] (0,0) grid (50,50);
		% BOX
		\draw (20,-10) rectangle (37,25) node[midway]{ckt\_e};
		% INPUTS
		\small
		\node (hide) at (0,30) {}; % just to expand background
		\node (a) at (20,-2.5) {}; % this is the reference point
		\draw [latex-] ($(a)+(0,25)$) -- ++(-10,0) node[left]{RAM\_CS};
		\draw [latex-] ($(a)+(0,20)$) -- ++(-10,0) node[left]{RAM\_WE};
		\draw [latex-] ($(a)+(0,15)$) -- ++(-10,0) node[left]{RAM\_OE};
		\draw [latex-] ($(a)+(0,10)$) -- ++(-10,0) node[left]{SEL\_OP1}
		node[pos=0.4,above]{4} node[pos=0.7]{/};
		\draw [latex-] ($(a)+(0,5)$) -- ++(-10,0) node[left]{SEL\_OP2}
		node[pos=0.4,above]{4} node[pos=0.7]{/};
		\draw [latex-] ($(a)+(0,0)$) -- ++(-10,0) node[left]{RAM\_DATA\_IN}
		node[pos=0.4,above]{8} node[pos=0.7]{/};
		\draw [latex-] ($(a)+(0,-5)$) -- ++(-10,0) node[left]{RAM\_ADDR\_IN}
		node[pos=0.4,above]{10} node[pos=0.7]{/};
		% OUTPUTS
		\draw [-latex] ($(a)+(17,10)$) -- ++(10,0) node[right]{RAM\_DATA\_OUT} node[pos=0.6,above]{8} node[pos=0.4]{/};
		\end{tikzpicture}
	\end{minipage}

	\item a) Line 4 is missing a semicolon, and the semicolon in line 5 should be outside of the parenthesis. Correction:
	
	\begin{minipage}{1\linewidth}
		\begin{lstlisting}[]
ENTITY ckt_a IS
	PORT (
		J,K	:	IN	STD_LOGIC;
		CLK	:	IN	STD_LOGIC;
		Q	:	OUT	STD_LOGIC);
END ckt_a;
		\end{lstlisting}
	\end{minipage}
		
	b) Line 5 should have double parentheses before the semicolon. \mbox{Correction:}
	
	\begin{minipage}{1\linewidth}
		\begin{lstlisting}[]
ENTITY ckt_b IS
	PORT (
		mr_fluffy	:	IN	STD_LOGIC_VECTOR(15 DOWNTO 0);
		mux_ctrl	:	IN	STD_LOGIC_VECTOR(3 DOWNTO 0);
		byte_out	:	OUT	STD_LOGIC_VECTOR(3 DOWNTO 0));
END ckt_b;
		\end{lstlisting}
	\end{minipage}
	
\end{enumerate}

\section*{Chapter 4}
\begin{enumerate}
	\begin{minipage}{1\linewidth}
	\item a)
		\begin{lstlisting}[]
LIBRARY IEEE;
USE IEEE.STD_LOGIC_1164.ALL;

ENTITY ckt_a IS
	PORT (
		A,B	:	IN	STD_LOGIC;
		F	:	OUT	STD_LOGIC);
END ckt_a;

ARCHITECTURE ckt_a_arc OF ckt_a IS
BEGIN
	F <= (NOT A AND B) OR A OR (A AND NOT B);
END ckt_a_arc;
		\end{lstlisting}
	\end{minipage}
	
	\begin{minipage}{1\linewidth}
		
		b)
		\begin{lstlisting}[]
LIBRARY IEEE;
USE IEEE.STD_LOGIC_1164.ALL;

ENTITY ckt_b IS
	PORT(
		A,B,C,D	:	IN	STD_LOGIC;
		F		:	OUT	STD_LOGIC);
END ckt_b;

ARCHITECTURE ckt_b_arc OF ckt_b IS
BEGIN
	F <= (NOT A AND C AND NOT D) OR (NOT B AND C) OR (B AND C AND NOT D);
END ckt_b_arc;
		\end{lstlisting}
	\end{minipage}
	
	\begin{minipage}{1\linewidth}
		c)
		\begin{lstlisting}[]
LIBRARY IEEE;
USE IEEE.STD_LOGIC_1164.ALL;

ENTITY ckt_c IS
	PORT(
		A,B,C,D	:	IN	STD_LOGIC;
		F		:	OUT	STD_LOGIC);
END ckt_c;

ARCHITECTURE ckt_c_arc OF ckt_c IS
	SIGNAL AND1,AND2,AND3	:	STD_LOGIC;
BEGIN
	AND1 <= NOT A OR B;
	AND2 <= NOT B OR C OR NOT D;
	AND3 <= NOT A OR D;
	F <= AND1 AND AND2 AND AND3;
END ckt_c_arc;
		\end{lstlisting}
	\end{minipage}
	
	\begin{minipage}{1\linewidth}
		d)
		\begin{lstlisting}[]
LIBRARY IEEE;
USE IEE.STD_LOGIC_1164.ALL;

ENTITY ckt_d IS
	PORT(
		A,B,C,D	:	IN	STD_LOGIC;
		F		:	OUT	STD_LOGIC);
END ckt_d;

ARCHITECTURE ckt_d_arc OF ckt_d IS
	SIGNAL AND1,AND2	:	STD_LOGIC;
BEGIN
	AND1 <= A OR B OR NOT C OR NOT D;
	AND2 <= A OR B OR C OR NOT D;
	F <= AND1 AND AND2;
END ckt_d_arc;
		\end{lstlisting}
	\end{minipage}

	\begin{minipage}{1\linewidth}
		e)
		\begin{lstlisting}[]
LIBRARY IEEE;
USE IEEE.STD_LOGIC_1164.ALL;

ENTITY ckt_e IS
	PORT(
		A,B,C	:	IN	STD_LOGIC;
		F		:	OUT	STD_LOGIC);
END ckt_e;

ARCHITECTURE ckt_e_arc OF ckt_e IS
	SIGNAL AND1,AND2,AND3,AND4	:	STD_LOGIC;
BEGIN
	AND1 <= NOT C OR B OR NOT A;
	AND2 <= C OR B OR NOT A;
	AND3 <= NOT C OR B OR A;
	AND4 <= C OR NOT B OR NOT A;
	F <= AND1 AND AND2 AND AND3 AND AND4;
END ckt_e_arc;
		\end{lstlisting}
	\end{minipage}

	\begin{minipage}{1\linewidth}
		f)
		\begin{lstlisting}[]
LIBRARY IEEE;
USE IEEE.STD_LOGIC_1164.ALL;

ENTITY ckt_f IS
	PORT(
		A,B,C,D	:	IN	STD_LOGIC;
		F		:	OUT	STD_LOGIC);
END ckt_f;

ARCHITECTURE ckt_f_arc OF ckt_f IS
	SIGNAL OR1,OR2	:	STD_LOGIC;
BEGIN
	OR1 <= NOT A AND NOT B AND NOT C AND D;
	OR2 <= NOT A AND NOT B AND C AND NOT D;
	F <= OR1 OR OR2;
END ckt_f_arc;
		\end{lstlisting}
	\end{minipage}
	
	\begin{minipage}{1\linewidth}
	\item a)
		\begin{lstlisting}
LIBRARY IEEE;
USE IEEE.STD_LOGIC_1164.ALL;

ENTITY ckt_a IS
	PORT(
		A,B,C,D	:	IN	STD_LOGIC;
		F		:	OUT	STD_LOGIC);
END ckt_a;

ARCHITECTURE ckt_a_conditional OF ckt_a IS
BEGIN
	F <=
		'1' WHEN( A = '0' AND C = '1' AND D = '0' ) ELSE
		'1' WHEN( B = '0' AND C = '1') ELSE
		'1' WHEN( B = '1' AND C = '1' AND D = '0' ) ELSE
		'0';
END ckt_a_conditional;

ARCHITECTURE ckt_a_selected OF ckt_a IS
	SIGNAL ins	:	STD_LOGIC_VECTOR(0 TO 3);
BEGIN
	ins <= A & B & C & D;
	WITH ins SELECT
		F <=
			'1' WHEN "0010"|"0110",
			'1' WHEN "0010"|"0011"|"1010"|"1011",
			'1' WHEN "0110"|"1110",
			'0' WHEN OTHERS;			
END ckt_a_selected;
		\end{lstlisting}
	\end{minipage}

	\begin{minipage}{1\linewidth}
		b)
		\begin{lstlisting}
LIBRARY IEEE;
USE IEEE.STD_LOGIC_1164.ALL;

ENTITY ckt_b IS
	PORT(
		A,B,C,D	:	IN	STD_LOGIC;
		F		:	OUT	STD_LOGIC);
END ckt_b;

ARCHITECTURE ckt_b_conditional OF ckt_b IS
BEGIN
	F <=
		'1' WHEN( ( A = '0' OR B = '1') AND ( B = '1' OR C = '1' OR D = '0' ) AND ( A = '0' OR D = '1') ) ELSE
		'0';
END ckt_b_conditional;

ARCHITECTURE ckt_b_selected OF ckt_b IS
	SIGNAL ins	:	STD_LOGIC_VECTOR(0 TO 3);
BEGIN
	ins <= A & B & C & D;
	WITH ins SELECT
		F <=
		'1' WHEN "0000"|"0001"|"0010"|"0011"|"0100",
		'1' WHEN "0110"|"0111",
		'1' WHEN "1111",
		'0' WHEN OTHERS;
END ckt_b_selected;
		\end{lstlisting}
	\end{minipage}

	\begin{minipage}{1\linewidth}
		c)
		\begin{lstlisting}
LIBRARY IEEE;
USE IEEE.STD_LOGIC_1164.ALL;

ENTITY ckt_c IS
	PORT(
		A,B,C,D	:	IN	STD_LOGIC;
		F		:	OUT	STD_LOGIC);
END ckt_c;

ARCHITECTURE ckt_c_conditional OF ckt_c IS
BEGIN
	F <=
		'1' WHEN( ( A = '1' OR B = '0' OR C = '0' OR D = '0') AND ( A = '1' OR B = '1' OR C = '1' OR D = '0' ) ) ELSE
		'0';
END ckt_c_conditional;

ARCHITECTURE ckt_c_selected OF ckt_c IS
	SIGNAL ins	:	STD_LOGIC_VECTOR(0 TO 3);
BEGIN
	ins <= A & B & C & D;
	WITH ins SELECT
		F <=
			'0' WHEN "0001"|"0011",
			'1' WHEN OTHERS;
END ckt_c_selected;
		\end{lstlisting}
	\end{minipage}

	\begin{minipage}{1\linewidth}
		d)
		\begin{lstlisting}
LIBRARY IEEE;
USE IEEE.STD_LOGIC_1164.ALL;

ENTITY ckt_d IS
	PORT(
		A,B,C,D	:	IN	STD_LOGIC;
		F		:	OUT	STD_LOGIC);
END ckt_d;

ARCHITECTURE ckt_d_conditional OF ckt_d IS
BEGIN
	F <=
		'1' WHEN( ( A = '0' AND B = '0' AND C = '0' AND D = '1') OR ( A = '0' AND B = '0' AND C = '1' AND D = '0' ) ) ELSE
		'0';
END ckt_d_conditional;

ARCHITECTURE ckt_d_selected OF ckt_d IS
	SIGNAL ins	:	STD_LOGIC_VECTOR(0 TO 3);
BEGIN
	ins <= A & B & C & D;
	WITH ins SELECT
		F <=
			'1' WHEN "0001"|"0010",
			'0' WHEN OTHERS;
END ckt_d_selected;
		\end{lstlisting}
	\end{minipage}
	
	\begin{minipage}{1\linewidth}
		\item
		\begin{lstlisting}
LIBRARY IEEE;
USE IEEE.STD_LOGIC_1164.ALL;

ENTITY and8 IS
	PORT(
		in0,in1,in2,in3,in4,in5,in6,in7	:	IN	STD_LOGIC;
		out1							:	OUT	STD_LOGIC);
END and8;

ARCHITECTURE and8_concurrent OF and8 IS
BEGIN
	out1 <= in0 AND in1 AND in2 AND in3 AND in4 AND in5 AND in6 AND in7;
END and8_concurrent;

ARCHITECTURE and8_conditional OF and8 IS
BEGIN
	out1 <= '1' WHEN (in0 = '1' AND in1 = '1' AND in2 = '1' AND in3 = '1' AND in4 = '1' AND in5 = '1' AND in6 = '1' AND in7 = '1') ELSE
	'0';
END and8_conditional;

ARCHITECTURE and8_selected OF and8 IS
	SIGNAL ins	:	STD_LOGIC_VECTOR(0 TO 7);
BEGIN
	ins <= in0 & in1 & in2 & in3 & in4 & in5 & in6 & in7;
	WITH ins SELECT
		out1 <=
			'1' WHEN "11111111",
			'0' WHEN OTHERS;
END and8_selected;
		\end{lstlisting}
	\end{minipage}

	\begin{minipage}{1\linewidth}
		\item
		\begin{lstlisting}
LIBRARY IEEE;
USE IEEE_STD_LOGIC_1164.ALL;

ENTITY or8 IS
	PORT(
		ins		:	IN	STD_LOGIC_VECTOR(7 DOWNTO 0);
		out1	:	OUT	STD_LOGIC);
END or8;

ARCHITECTURE or8_concurrent OF or8 IS
BEGIN
	out1 <= ins(1) OR ins(2) OR ins(3) OR ins(4) OR ins(5) OR ins(6) OR ins(7);
END or8_concurrent;

ARCHITECTURE or8_conditional OF or8 IS
BEGIN
	out1 <= '1' WHEN (ins(1) = '1' OR ins(2) = '1' OR ins(3) = '1' OR ins(4) = '1' OR ins(5) = '1' OR ins(6) = '1' OR ins(7) = '1') ELSE
	'0';
END or8_conditional;

ARCHITECTURE or8_selected OF or8 IS
BEGIN
	WITH ins SELECT
		out1 <=
			'0' WHEN "00000000",
			'1' WHEN OTHERS;
END or8_selected;
		\end{lstlisting}
	\end{minipage}

	\begin{minipage}{1\linewidth}
		\item 
		\begin{lstlisting}
LIBRARY IEEE;
USE IEEE.STD_LOGIC_1164.ALL;

ENTITY mux8to1 IS
	PORT(
		inputs	:	IN	STD_LOGIC_VECTOR(7 DOWNTO 0);		
		sel		:	IN	STD_LOGIC_VECTOR(2 DOWNTO 0);
		output	:	OUT	STD_LOGIC);
END mux8to1;

ARCHITECTURE mux8to1_conditional OF mux8to1 IS
BEGIN
	output <=
		inputs(0) WHEN (sel = "000") ELSE
		inputs(1) WHEN (sel = "001") ELSE
		inputs(2) WHEN (sel = "010") ELSE
		inputs(3) WHEN (sel = "011") ELSE
		inputs(4) WHEN (sel = "100") ELSE
		inputs(5) WHEN (sel = "101") ELSE
		inputs(6) WHEN (sel = "110") ELSE
		inputs(7) WHEN (sel = "111") ELSE
		'0';
END mux8to1_conditional;

ARCHITECTURE mux8to1_selected OF mu8to1 IS
BEGIN
	WITH sel SELECT
		output <=
			inputs(0) WHEN "000",
			inputs(1) WHEN "001",
			inputs(2) WHEN "010",
			inputs(3) WHEN "011",
			inputs(4) WHEN "100",
			inputs(5) WHEN "101",
			inputs(6) WHEN "110",
			inputs(7) WHEN "111",
			'0' WHEN OTHERS;
END mux8to1_selected;
		\end{lstlisting}
	\end{minipage}

	\begin{minipage}{1\linewidth}
		\item
		\begin{lstlisting}
LIBRARY IEEE;
USE IEEE.STD_LOGIC_1164.ALL;

ENTITY dec3to8_ActiveHigh IS
	PORT (
		inputs	:	IN	STD_LOGIC_VECTOR(2 DOWNTO 0);
		outputs	:	OUT	STD_LOGIC_VECTOR(0 TO 7));
END dec3to8_ActiveHigh;

ARCHITECTURE dec3to8_ActiveHigh_conditional OF dec3to8_ActiveHigh IS
BEGIN
	outputs <=
		"10000000" WHEN (inputs = "000") ELSE
		"01000000" WHEN (inputs = "001") ELSE
		"00100000" WHEN (inputs = "010") ELSE
		"00010000" WHEN (inputs = "011") ELSE
		"00001000" WHEN (inputs = "100") ELSE
		"00000100" WHEN (inputs = "101") ELSE
		"00000010" WHEN (inputs = "110") ELSE
		"00000001" WHEN (inputs = "111") ELSE
		"00000000";
END dec3to8_ActiveHigh_conditional;

ARCHITECTURE dec3to8_ActiveHigh_selected OF dec3to8_ActiveHigh IS
BEGIN
	WITH inputs SELECT
		outputs <=
			"10000000" WHEN "000",
			"01000000" WHEN "001",
			"00100000" WHEN "010",
			"00010000" WHEN "011",
			"00001000" WHEN "100",
			"00000100" WHEN "101",
			"00000010" WHEN "110",
			"00000001" WHEN "111",
			"00000000" WHEN OTHERS;
END dec3to8_ActiveHigh_selected;
		\end{lstlisting}
	\end{minipage}

	\begin{minipage}{1\linewidth}
		\item 
		\begin{lstlisting}
LIBRARY IEEE;
USE IEEE.STD_LOGIC_1164.ALL;

ENTITY dec3to8_ActiveLow IS
	PORT (
		inputs	:	IN	STD_LOGIC_VECTOR(2 DOWNTO 0);
		outputs	:	OUT	STD_LOGIC_VECTOR(0 TO 7));
END dec3to8_ActiveLow;

ARCHITECTURE dec3to8_ActiveLow_conditional OF dec3to8_ActiveLow IS
BEGIN
	outputs <=
		"01111111" WHEN (inputs = "000") ELSE
		"10111111" WHEN (inputs = "001") ELSE
		"11011111" WHEN (inputs = "010") ELSE
		"11101111" WHEN (inputs = "011") ELSE
		"11110111" WHEN (inputs = "100") ELSE
		"11111011" WHEN (inputs = "101") ELSE
		"11111101" WHEN (inputs = "110") ELSE
		"11111110" WHEN (inputs = "111") ELSE
		"11111111";
END dec3to8_ActiveLow_conditional;

ARCHITECTURE dec3to8_ActiveLow_selected OF dec3to8_ActiveLow IS
BEGIN
	WITH inputs SELECT
		outputs <=
			"01111111" WHEN "000",
			"10111111" WHEN "001",
			"11011111" WHEN "010",
			"11101111" WHEN "011",
			"11110111" WHEN "100",
			"11111011" WHEN "101",
			"11111101" WHEN "110",
			"11111110" WHEN "111",
			"11111111" WHEN OTHERS;
END dec3to8_ActiveLow_selected;
		\end{lstlisting}
	\end{minipage}
\end{enumerate}

\section*{Chapter 5}
\begin{enumerate}
	\item a)
	\begin{lstlisting}
LIBRARY IEEE;
USE IEEE.STD_LOGIC_1164.ALL;

ENTITY ckt_a IS
	PORT(
		A,B	:	IN	STD_LOGIC;
		F	:	OUT	STD_LOGIC);
END ckt_a;

ARCHITECTURE ckt_a_case OF ckt_a IS
	SIGNAL ins	:	STD_LOGIC_VECTOR(0 TO 1);
BEGIN
	ins <= A & B;
	logic_process: PROCESS(ins)
	BEGIN
		CASE (ins) IS
			WHEN "01" =>
				F	<=	'1';
			WHEN "10" =>
				F	<=	'1';
			WHEN "11" =>
				F	<=	'1';
			WHEN OTHERS =>
				F	<=	'0';
		END CASE;
	END PROCESS logic_process;
END ckt_a_case;

ARCHITECTURE ckt_a_if OF ckt_a IS
BEGIN
	logic_process: PROCESS(A,B)
	BEGIN
		IF (A = '0' AND B = '1') THEN
			F	<=	'1';
		ELSIF (A = '1') THEN
			F	<=	'1';
		ELSE
			F	<=	'0';
		END IF;
	END PROCESS logic_process;
END ckt_a_if;
	\end{lstlisting}

		b)
		\begin{lstlisting}
LIBRARY IEEE;
USE IEEE.STD_LOGIC_1164.ALL;

ENTITY ckt_b IS
	PORT(
		A,B,C,D	:	IN	STD_LOGIC;
		F		:	OUT	STD_LOGIC);
END ckt_b;

ARCHITECTURE ckt_b_case OF ckt_b IS
	SIGNAL ins	:	STD_LOGIC_VECTOR(0 TO 3);
BEGIN
	ins <= A & B & C & D;
	logic_process:	PROCESS(ins)
	BEGIN
		CASE (ins) IS
			WHEN "0010" =>
				F	<=	'1';
			WHEN "0110" =>
				F	<=	'1';
			WHEN "0011" =>
				F	<=	'1';
			WHEN "1010" =>
				F	<=	'1';
			WHEN "1011" =>
				F	<=	'1';
			WHEN "1110" =>
				F	<=	'1';
			WHEN OTHERS =>
				F	<= '0';
		END CASE;
	END PROCESS logic_process;
END ckt_b_case;

ARCHITECTURE ckt_b_if OF ckt_b IS
BEGIN
	logic_process:	PROCESS(A,B,C,D)
	BEGIN
		IF ( A = '0' AND C = '1' AND D = '1' ) THEN
			F	<=	'1';
		ELSIF ( B = '0' AND C = '1' ) THEN
			F	<= '1';
		ELSIF ( B = '1' AND C = '1' AND D = '0' ) THEN
			F	<=	'1';
		ELSE
			F	<=	'0';
		END IF;
	END PROCESS logic_process;
END ckt_b_if;
		\end{lstlisting}

		c)
		\begin{lstlisting}
LIBRARY IEEE;
USE IEEE.STD_LOGIC_1164.ALL;

ENTITY ckt_c IS
	PORT(
		A,B,C,D	:	IN	STD_LOGIC;
		F		:	OUT	STD_LOGIC);
END ckt_c;

ARCHITECTURE ckt_c_case OF ckt_c IS
	SIGNAL ins	:	STD_LOGIC_VECTOR(0 TO 3);
BEGIN
	ins	<=	A & B & C & D;
	logic_process:	PROCESS(ins)
	BEGIN
		CASE (ins) IS
			WHEN "0000" =>
				F	<=	'1';
			WHEN "0001" =>
				F	<=	'1';
			WHEN "0010" =>
				F	<=	'1';
			WHEN "0011" =>
				F	<=	'1';
			WHEN "0100" =>
				F	<=	'1';
			WHEN "0110" =>
				F	<=	'1';
			WHEN "0111" =>
				F	<=	'1';
			WHEN "1111" =>
				F	<=	'1';
			WHEN OTHERS =>
				F	<=	'0';
		END CASE;
	END PROCESS logic_process;
END ckt_c_case;

ARCHITECTURE ckt_c_if OF ckt_c IS
BEGIN
	logic_process:	PROCESS(A,B,C,D)
	BEGIN
		IF ( ( A = '0' OR B = '1' ) AND ( B = '0' OR C = '1' OR D = '0' ) AND ( A = '0' OR D = '1' ) ) THEN
			F	<=	'1';
		ELSE
			F	<=	'0';
		END IF;
	END PROCESS logic_process;
END ckt_c_if;
		\end{lstlisting}	

		d)
		\begin{lstlisting}
LIBRARY IEEE;
USE IEEE.STD_LOGIC_1164.ALL;

ENTITY ckt_d IS
	PORT(
		A,B,C,D	:	IN	STD_LOGIC;
		F		:	OUT	STD_LOGIC);
END ckt_d;

ARCHITECTURE ckt_d_case OF ckt_d IS
	SIGNAL ins	:	STD_LOGIC_VECTOR(0 TO 3);
BEGIN
	ins	<=	A & B & C & D;
	logic_process:	PROCESS(ins)
	BEGIN
		CASE (ins) IS
			WHEN "0001" =>
				F	<=	'0';
			WHEN "0011" =>
				F	<=	'0';
			WHEN "0100" =>
				F	<=	'0';
			WHEN "0101" =>
				F	<=	'0';
			WHEN OTHERS =>
				F	<=	'1';
		END CASE;
	END PROCESS logic_process;
END ckt_d_case;

ARCHITECTURE ckt_d_if OF ckt_d IS
BEGIN
	logic_process:	PROCESS(A,B,C,D)
	BEGIN
		IF ( A = '0' AND B = '0' AND C = '0' AND D = '1' ) THEN
			F	<=	'0';
		ELSIF ( A = '1' AND B = '1' AND C = '1' AND D = '1' ) THEN
			F	<=	'0';
		ELSIF ( A = '0' AND B = '1' AND C = '0' AND D = '0' ) THEN
			F	<=	'0';
		ELSIF ( A = '0' AND B = '1' AND C = '0' AND D = '1' ) THEN
			F	<=	'0';
		ELSE
			F	<=	'1';
		END IF;
	END PROCESS logic_process;
END ckt_d_if;
		\end{lstlisting}

		e)
		\begin{lstlisting}
LIBRARY IEEE;
USE IEEE.STD_LOGIC_1164.ALL;

ENTITY ckt_e IS
	PORT(
		A,B,C,D	:	IN	STD_LOGIC;
		F		:	OUT	STD_LOGIC);
END ckt_e;

ARCHITECTURE ckt_e_case OF ckt_e IS
	SIGNAL ins	:	STD_LOGIC_VECTOR(0 TO 3);
BEGIN
	ins	<=	A & B & C & D;
	logic_process:	PROCESS(ins)
	BEGIN
		CASE (ins) IS
			WHEN "0001" =>
				F	<=	'1';
			WHEN "0010" =>
				F	<=	'1';
			WHEN OTHERS =>
				F	<=	'0';
		END CASE;
	END PROCESS logic_process;
END ckt_e_case;

ARCHITECTURE ckt_e_if OF ckt_e IS
BEGIN
	logic_process:	PROCESS(A,B,C,D)
	BEGIN
		IF ( A = '0' AND B = '0' AND C = '0' AND D = '1' ) THEN
			F	<=	'1';
		ELSIF ( A = '0' AND B = '0' AND C = '1' AND D = '0' ) THEN
			F	<=	'1';
		ELSE
			F	<=	'0';
		END IF;
	END PROCESS logic_process;
END ckt_e_if;
		\end{lstlisting}

	\item
		\begin{lstlisting}
LIBRARY IEEE;
USE IEEE.STD_LOGIC_1164.ALL;

ENTITY Exercise_5_2 IS
	PORT(
		A,B		:	IN	STD_LOGIC_VECTOR(1 DOWNTO 0);
		D		:	IN	STD_LOGIC;
		E_out	:	OUT	STD_LOGIC);
END Exercise_5_2;

ARCHITECTURE Exercise_5_2_case OF Exercise_5_2 IS
	SIGNAL	Aout,Bout,Cout,Dout	:	STD_LOGIC;
	SIGNAL	Cin					:	STD_LOGIC_VECTOR(0 TO 1);
	SIGNAL	Ein					:	STD_LOGIC_VECTOR(0 TO 2);
BEGIN
	
	Cin	<=	B(2) & Dout;
	Ein	<=	Aout & Bout & Cout;
	
	devA:	PROCESS(A)
	BEGIN
		CASE (A) IS
			WHEN "00"|"01"|"10" =>
				Aout	<=	'0';
			WHEN "11" =>
				Aout	<=	'1';
		END CASE;
	END PROCESS devA;
	
	devB:	PROCESS(B)
	BEGIN
		CASE(B) IS
			WHEN "00" =>
				Bout	<=	'0';
			WHEN "01"|"10"|"11" =>
				Bout	<=	'1';
		END CASE;
	END PROCESS devB;
	
	devC:	PROCESS(Cin)
	BEGIN
		CASE(Cin) IS
			WHEN "00"|"01"|"10" =>
				Cout	<=	'0';
			WHEN "11" =>
				Cout	<=	'1';
		END CASE;
	END PROCESS devC;
	
	devD:	PROCESS(D)
	BEGIN
		CASE(D) IS
			WHEN '0' =>
				Dout	<=	'1';
			WHEN '1' =>
				Dout	<=	'0';
		END CASE;
	END PROCESS devD;
	
	devE:	PROCESS(Ein) IS
	BEGIN
		CASE (Ein) IS
			WHEN "000" =>
				E_out	<=	'0';
			WHEN OTHERS =>
				E_out	<=	'1';
		END CASE;
	END PROCESS devE;
		
END Exercise_5_2_case;

ARCHITECTURE Exercise_5_2_if OF Exercise_5_2 IS
BEGIN
	logic_process:	PROCESS(A,B,D) IS
	BEGIN
		IF ( A = "11") THEN
			E_out	<=	'1';
		ELSIF ( B = "01"|"10"|"11" ) THEN
			E_out	<=	'1';
		ELSIF ( B(2) = '1' AND D = '0' ) THEN
			E_out	<=	'1';
		ELSE
			E_out	<=	'0';
		END IF;
	END PROCESS logic_process;
END Exercise_5_2_if;
		\end{lstlisting}
		
	\item
	\begin{lstlisting}
LIBRARY IEEE;
USE IEEE.STD_LOGIC_1164.ALL;

ENTITY Exercise_5_3 IS
	PORT(
		A,B		:	IN	STD_LOGIC_VECTOR(1 DOWNTO 0);
		D		:	IN	STD_LOGIC;
		E_out	:	OUT	STD_LOGIC);
END Exercise_5_3;

ARCHITECTURE Exercise_5_3_arc OF Exercise_5_3 IS
	SIGNAL Aout,Bout,Cout,Dout	:	STD_LOGIC;
BEGIN
	Aout	<=	A(1) AND A(2);
	Bout	<=	B(1) OR B(2);
	Cout	<=	B(2) AND Dout;
	Dout	<=	NOT D;
	E_out	<=	Aout OR Bout OR Cout;
END Exercise_5_3_arc;
	\end{lstlisting}
	
	\item 
	\begin{lstlisting}
LIBRARY IEEE;
USE IEEE.STD_LOGIC_1164.ALL;

ENTITY and8 IS
	PORT(
		ins		:	IN	STD_LOGIC_VECTOR(7 DOWNTO 0);
		out1	:	OUT	STD_LOGIC);
END and8;

ARCHITECTURE and8_arc OF and8 IS
BEGIN
	logic_process:	PROCESS(ins)
	BEGIN
		IF ( ins = "11111111" ) THEN
			out1	<=	'1';
		ELSE
			out1	<=	'0';
		END IF;
	END PROCESS logic_process;
END and8_arc;
	\end{lstlisting}
	
	\item
	\begin{lstlisting}
LIBRARY IEEE;
USE IEEE.STD_LOGIC_1164.ALL;

ENTITY or8 IS
	PORT(
		ins		:	IN	STD_LOGIC_VECTOR(7 DOWNTO 0);
		out1	:	OUT	STD_LOGIC);
END or8;

ARCHITECTURE or8_arc OF or8 IS
BEGIN
	logic_process:	PROCESS(ins)
	BEGIN
		IF ( ins = "00000000" ) THEN
			out1	<=	'0';
		ELSE
			out1	<=	'1';
		END IF;
	END PROCESS logic_process;
END or8_arc;
	\end{lstlisting}
	
	\item 
	\begin{lstlisting}
LIBRARY IEEE;
USE IEEE.STD_LOGIC_1164.ALL;

ENTITY mux8to1 IS
	PORT(
		ins		:	IN	STD_LOGIC_VECTOR(7 DOWNTO 0);
		sel		:	IN	STD_LOGIC_VECTOR(2 DOWNTO 0);
		out1	:	OUT	STD_LOGIC);
END mux8to1;

ARCHITECTURE mux8to1_if OF mux8to1 IS
BEGIN
	logic_process:	PROCESS(ins,sel) IS
	BEGIN
		IF ( sel = "000" ) THEN
			out1	<=	ins(0);
		ELSIF ( sel = "001" ) THEN
			out1	<=	ins(1);
		ELSIF ( sel = "010" ) THEN
			out1	<=	ins(2);
		ELSIF ( sel = "011" ) THEN
			out1	<=	ins(3);
		ELSIF ( sel = "100" ) THEN
			out1	<=	ins(4);
		ELSIF ( sel = "101" ) THEN
			out1	<=	ins(5);
		ELSIF ( sel = "110" ) THEN
			out1	<=	ins(6);
		ELSIF ( sel = "111" ) THEN
			out1	<=	ins(7);
		ELSE
			out1	<=	'0';
		END IF;			
	END PROCESS logic_process;
END mux8to1_if;

ARCHITECTURE mux8to1_cases OF mux8to1 IS
BEGIN
	logic_process:	PROCESS(ins,sel) IS
	BEGIN
		CASE (sel) IS
			WHEN "000" =>
				out1	<= ins(0);
			WHEN "001" =>
				out1	<= ins(1);
			WHEN "010" =>
				out1	<= ins(2);
			WHEN "011" =>
				out1	<= ins(3);
			WHEN "100" =>
				out1	<= ins(4);
			WHEN "101" =>
				out1	<= ins(5);
			WHEN "110" =>
				out1	<= ins(6);
			WHEN "111" =>
				out1	<= ins(7);
			WHEN OTHERS =>
				out1	<=	'0';
		END CASE;
	END PROCESS logic_process;
END mux8to1_cases;
	\end{lstlisting}
	
	\item
	\begin{lstlisting}
LIBRARY IEEE;
USE IEEE.STD_LOGIC_1164.ALL;

ENTITY dec3to8_ActiveLow IS
	PORT(
		ins		:	IN	STD_LOGIC_VECTOR(2 DOWNTO 0);
		outs	:	OUT	STD_LOGIC_VECTOR(7 DOWNTO 0));
END dec3to8_ActiveLow;

ARCHITECTURE dec3to8_ActiveLow_if OF dec3to8_ActiveLow IS
BEGIN
	logic_process:	PROCESS(ins) IS
	BEGIN
		IF ( ins = "000" ) THEN
			outs	<=	"01111111";
		ELSIF ( ins = "001" ) THEN
			outs	<=	"10111111";
		ELSIF ( ins = "010" ) THEN
			outs	<=	"11011111";
		ELSIF ( ins = "011" ) THEN
			outs	<=	"11101111";
		ELSIF ( ins = "100" ) THEN
			outs	<=	"11110111";
		ELSIF ( ins = "101" ) THEN
			outs	<=	"11111011";
		ELSIF ( ins = "110" ) THEN
			outs	<=	"11111101";
		ELSIF ( ins = "111" ) THEN
			outs	<=	"11111110";
		ELSE
			outs	<=	"11111111";
		END IF;
	END PROCESS logic_process;
END dec3to8_ActiveLow_if;

ARCHITECTURE dec3to8_ActiveLow_cases OF dec3to8_ActiveLow IS
BEGIN
	logic_process:	PROCESS(ins) IS
	BEGIN
		CASE (ins) IS
			WHEN "000" =>
				outs	<=	"01111111";
			WHEN "001" =>
				outs	<=	"10111111";
			WHEN "010" =>
				outs	<=	"11011111";
			WHEN "011" =>
				outs	<=	"11101111";
			WHEN "100" =>
				outs	<=	"11110111";
			WHEN "101" =>
				outs	<=	"11111011";
			WHEN "110" =>
				outs	<=	"11111101";
			WHEN "111" =>
				outs	<=	"11111110";
			WHEN OTHERS =>
				outs	<=	"11111111";
		END CASE;
	END PROCESS logic_process;
END dec3to8_ActiveLow_cases;
	\end{lstlisting}
\end{enumerate}

\setcounter{section}{7}
\section*{Chapter 7}
\begin{enumerate}
	\item 
	\begin{lstlisting}
LIBRARY IEEE;
USE ieee.std_logic_1164.all;

ENTITY dff_1 IS
	PORT(
		S,D,CLK,R	:	IN	STD_LOGIC;
		Q,NOTQ		:	OUT	STD_LOGIC);
END dff_1;

ARCHITECTURE dff_1_arc OF dff_1 IS
	SIGNAL q_temp	:	STD_LOGIC;
BEGIN
	dff_process: PROCESS(CLK,S,R) IS
	BEGIN
		IF (S = '0') THEN
			q_temp <= '1';
		ELSIF (R = '0') THEN
			q_temp <= '0';
		ELSIF (RISING_EDGE(CLK)) THEN
			q_temp <= D;
		END IF;
	END PROCESS dff_process;
	
	Q <= q_temp;
	NOTQ <= NOT q_temp;
END dff_1_arc;
	\end{lstlisting}
	\item See solution to Exercise 1, as the given solution works in both cases.
	
	\item 
	\begin{lstlisting}
LIBRARY IEEE;
USE IEEE.STD_LOGIC_1164.ALL;

ENTITY dff_3 IS
	PORT(
		S,D,CLK,R	:	IN	STD_LOGIC;
		Q,NOTQ		:	OUT	STD_LOGIC);
END dff_3;

ARCHITECTURE dff_3_arc OF dff_3 IS
	SIGNAL q_temp	:	STD_LOGIC;
BEGIN
	dff_process: PROCESS(CLK,S,R) IS
	BEGIN
		IF (RISING_EDGE(CLK)) THEN
			IF (S = '0') THEN
				q_temp <= '1';
			ELSIF (R = '0') THEN
				q_temp <= '0';
			ELSE
				q_temp <= D;
			END IF;
		END IF;
	END PROCESS dff_process;
	
	Q <= q_temp;
	NOTQ <= NOT q_temp;
END dff_3_arc;
	\end{lstlisting}
	
	\item 
	\begin{lstlisting}
LIBRARY IEEE;
USE IEEE.STD_LOGIC_1164.ALL;

ENTITY dff_4 IS
	PORT(
		S,D,CLK,R	:	IN	STD_LOGIC;
		Q,NOTQ		:	OUT	STD_LOGIC);
END dff_4;

ARCHITECTURE dff_4_arc OF dff_4 IS
	SIGNAL q_temp	:	STD_LOGIC;
BEGIN
	dff_process: PROCESS(CLK,S,R) IS
	BEGIN
		IF (S = '0' AND R = '0') THEN
			q_temp <= NOT q_temp;
		ELSIF (S = '0') THEN
			q_temp <= '1';
		ELSIF (R = '0') THEN
			q_temp <= '0';
		ELSIF (RISING_EDGE(CLK)) THEN
			q_temp <= D;
		END IF;
	END PROCESS dff_process;
	
	Q <= q_temp;
	NOTQ <= NOT q_temp;
END dff_4_arc;
	\end{lstlisting}
	
	\item \begin{lstlisting}
LIBRARY IEEE;
USE IEEE.STD_LOGIC_1164.ALL;

ENTITY tff IS
	PORT(
		S,T,CLK,R	:	IN	STD_LOGIC;
		Q,NOTQ		:	OUT	STD_LOGIC);
END tff;

ARCHITECTURE tff_equation OF tff IS
	SIGNAL q_temp	:	STD_LOGIC;
BEGIN
	tff_process:	PROCESS(S,R,CLK) IS
	BEGIN
		IF ( S = '0' ) THEN
			q_temp	<=	'1';
		ELSIF ( R = '0' ) THEN
			q_temp	<=	'0';
		ELSIF ( RISING_EDGE(CLK) ) THEN
			q_temp	<=	q_temp XOR T;
		END IF;
	END PROCESS tff_process;
	
	Q		<=	q_temp;
	NOTQ	<=	NOT q_temp;
END tff_equation;

ARCHITECTURE tff_behavioral OF tff IS
	SIGNAL q_temp	:	STD_LOGIC;
BEGIN
	tff_process:	PROCESS(S,R,CLK) IS
	BEGIN
		IF ( S = '0' ) THEN
			q_temp	<=	'1';
		ELSIF ( R = '0' ) THEN
			q_temp	<=	'0';
		ELSIF ( RISING_EDGE(CLK) AND T = '1' ) THEN
			q_temp	<=	NOT q_temp;
		END IF;
	END PROCESS tff_process;
	
	Q		<=	q_temp;
	NOTQ	<=	NOT q_temp;
END tff_behavioral;
	\end{lstlisting}
	
	\item See the second architecture of the previous solution (tff\_behavioral).
\end{enumerate}

\section*{Chapter 8}
\begin{enumerate}
	\item \begin{minipage}[t]{0.75\textwidth}
		\vspace{0pt}\raggedright
		\centering
		\begin{tikzpicture}[>=stealth',shorten >=1pt,auto,node distance=2cm]
		%FSM states
		\node[state with output] (ST0) {A \nodepart{lower} 0};
		\node[state with output] (ST1) [below right= of ST0] {B \nodepart{lower} 1};
		\node[state with output, initial, initial text=RESET] (ST2) [below left= of ST0] {C \nodepart{lower} 1};
		
		% FSM state changes/variables
		\path[->]  (ST0)  edge [loop above] node (helper) {\NOTT{X}/Z2} (ST0);
		\path[->]  (ST0)  edge [bend left=18]    node {X/\NOTT{Z2}} (ST1);
		\path[->]  (ST1)  edge [bend left=18]    node {X/Z2} (ST2);
		\path[->]  (ST1)  edge [bend left=18]    node[pos=0.85] {\NOTT{X}/\NOTT{Z2}} (ST0);
		\path[->]  (ST2)  edge [bend left=18]    node {X/\NOTT{Z2}} (ST0);
		\path[->]  (ST2)  edge [bend left=18]	 node[below] {\NOTT{X}/Z2} (ST1);
		
		% legend
		\node[state with output]  (legend) [left =of helper.north] {State \nodepart{lower} Z1};
		\node[below =.1cm of legend, align=left] (S) {input/outut(Z2)};
		\end{tikzpicture}
	\end{minipage}

	\item \begin{lstlisting}
LIBRARY IEEE;
USE IEEE.STD_LOGIC_1164.ALL;

ENTITY Exercise_8_2 IS
	PORT(
		CLK	:	IN	STD_LOGIC;
		X	:	IN	STD_LOGIC_VECTOR(1 TO 2);
		Y	:	OUT	STD_LOGIC_VECTOR(1 DOWNTO 0);
		Z	:	OUT	STD_LOGIC);
END Exercise_8_2;

ARCHITECTURE Exercise_8_2_arc OF Exercise_8_2 IS
	TYPE STATE_TYPE IS (ST10,ST01,ST11);
	SIGNAL PS,NS	:	STATE_TYPE;
BEGIN
	sync_process:	PROCESS(CLK) IS
	BEGIN
		IF(RISING_EDGE(CLK)) THEN
			PS	<=	NS;
		END IF;
	END PROCESS sync_process;
	
	comb_process:	PROCESS(X,PS) IS
	BEGIN
		Z	<=	'0';
		CASE (PS) IS
			WHEN ST10 =>
				IF ( X(1) = '0' ) THEN
					NS	<=	ST10;
					Z	<=	'0';
				ELSIF ( X(1) = '1' ) THEN
					NS	<=	ST01;
					Z	<=	'0';
				END IF;
			WHEN ST01 =>
				IF ( X(2) = '0' ) THEN
					NS	<=	ST10;
					Z	<=	'1';
				ELSIF ( X(2) = '1' ) THEN
					NS	<=	ST11;
					Z	<=	'0';
				END IF;
			WHEN ST11 =>
				IF ( X(2) = '0' ) THEN
					NS	<=	ST10;
					Z	<=	'1';
				ELSIF ( X(2) = '1' ) THEN
					NS	<=	ST11;
					Z	<=	'0';
				END IF;
			WHEN OTHERS =>
				NS	<=	ST10;
				Z	<=	'0';
		END CASE;
	END PROCESS comb_process;
	
		WITH PS SELECT
	Y	<=	"01" WHEN ST01,
			"10" WHEN ST10,
			"11" WHEN ST11,
			"00" WHEN OTHERS;
END Exercise_8_2_arc;
	\end{lstlisting}
	
	\item \begin{minipage}[t]{0.75\textwidth}
		\vspace{0pt}\raggedright
		\centering
		\begin{tikzpicture}[>=stealth',shorten >=1pt,auto,node distance=2cm]
		% FSM states
		\node[state with output] (ST1) {S1 \nodepart{lower} 0};
		\node[state with output] (ST2) [below right= of ST1] {S2 \nodepart{lower} 0};
		\node[state with output] (ST3) [below left= of ST1] {S3 \nodepart{lower} 1};
		
		% FSM state changes/variables
		\path[->]  (ST1)  edge [loop above] node (helper) {\NOTT{BUM1}/\NOTT{TOUT}} (ST1);
		\path[->]  (ST1)  edge [bend left=18]    node {BUM1/TOUT} (ST2);
		\path[->]  (ST2)  edge [bend left=18]    node {TOUT} (ST3);
		\path[->]  (ST3)  edge [bend left=18]    node {\NOTT{BUM2}/\NOTT{TOUT}} (ST2);
		\path[->]  (ST3)  edge [bend left=18]    node {BUM2/\NOTT{TOUT}} (ST1);
		
		% legend
		\node[state with output]  (legend) [left =of helper.north] {State \nodepart{lower} CTA};
		\node[below =.1cm of legend, align=left] (S) {input/outut};
		\node[below = .5cm of legend, align=left] (S) {output};
		\end{tikzpicture}
	\end{minipage}

	\item \begin{lstlisting}
LIBRARY IEEE;
USE IEEE.STD_LOGIC_1164.ALL;

ENTITY Exercise_8_4 IS
	PORT(
		CLK	:	IN	STD_LOGIC;
		X	:	IN	STD_LOGIC_VECTOR(1 TO 2);
		Z	:	OUT	STD_LOGIC_VECTOR(1 TO 2));
END Exercise_8_4;

ARCHITECTURE Exercise_8_4_arc OF Exercise_8_4 IS
	TYPE STATE_TYPE IS (ST00,ST01,ST10);
	SIGNAL PS,NS	:	STATE_TYPE;
BEGIN
	sync_process:	PROCESS(CLK) IS
	BEGIN
		IF(RISING_EDGE(CLK)) THEN
			PS	<=	NS;
		END IF;
	END PROCESS sync_process;
	
	comb_process:	PROCESS(X,PS) IS
	BEGIN
		Z	<=	"00";
		CASE (PS) IS
			WHEN ST00 =>
				Z(1)	<=	'0';
				IF ( X(1) = '0' ) THEN
					Z(2)	<=	'0';
					NS		<=	ST10;
				ELSIF ( X(1) = '1' ) THEN
					Z(2)	<=	'1';
					NS		<=	ST01;
				END IF;
			WHEN ST01 =>
				Z(1)	<=	'1';
				IF ( X(2) = '0' ) THEN
					Z(2)	<=	'1';
					NS		<=	ST10;
				ELSIF ( X(2) = '1' ) THEN
					Z(2)	<=	'0';
					NS		<=	ST00;
				END IF;
			WHEN ST10 =>
				Z(1)	<=	'1';
				IF ( X(1) = '0' ) THEN
					Z(2)	<=	'1';
					NS		<=	ST00;
				ELSIF ( X(1) = '1' ) THEN
					Z(2)	<=	'1';
					NS		<=	ST01;
				END IF;
			WHEN OTHERS =>
				Z	<=	"00";
				NS	<=	ST00;
		END CASE;
	END PROCESS comb_process;
END Exercise_8_4_arc;
	\end{lstlisting}
	
	\item \begin{minipage}[t]{0.75\textwidth}
		\vspace{0pt}\raggedright
		\centering
		\begin{tikzpicture}[>=stealth',shorten >=1pt,auto,node distance=2cm]
		% FSM states
		\node[state with output] (ST1) {A/001 \nodepart{lower} 0};
		\node[state with output] (ST2) [below right= of ST1] {B/010 \nodepart{lower} 1};
		\node[state with output, initial, initial text=RESET] (ST3) [below left= of ST1] {C/100 \nodepart{lower} 1};
		
		% FSM state changes/variables
		\path[->]  (ST1)  edge [loop above] node (helper) {0/1} (ST1);
		\path[->]  (ST1)  edge [bend left=18]    node {1/0} (ST2);
		\path[->]  (ST2)  edge [bend left=18]    node[pos=0.8] {0/0} (ST1);
		\path[->]  (ST2)  edge [bend left=18]    node {1/1} (ST3);
		\path[->]  (ST3)  edge [bend left=18]    node[below] {0/1} (ST2);
		\path[->]  (ST3)  edge [bend left=18]    node {1/0} (ST1);
		
		% legend
		\node[state with output]  (legend) [left =of helper.north] {State \nodepart{lower} Z1};
		\node[below =.1cm of legend, align=left] (S) {X/Z2};
		\end{tikzpicture}
	\end{minipage}

	\item \begin{lstlisting}
LIBRARY IEEE;
USE IEEE.STD_LOGIC_1164.ALL;

ENTITY Exercise_8_6 IS
	PORT(
		CLK,X	:	IN	STD_LOGIC;
		Y		:	OUT	STD_LOGIC_VECTOR(1 DOWNTO 0);
		Z		:	OUT	STD_LOGIC_VECTOR(1 TO 2));
END Exercise_8_6;

ARCHITECTURE Exercise_8_6_arc OF Exercise_8_6 IS
	TYPE STATE_TYPE IS (ST00,ST01,ST10,ST11);
	SIGNAL PS,NS	:	STATE_TYPE;
BEGIN
	sync_process:	PROCESS(CLK) IS
	BEGIN
		IF(RISING_EDGE(CLK)) THEN
			PS	<=	NS;
		END IF;
	END PROCESS sync_process;
	
	comb_process:	PROCESS(X,PS) IS
	BEGIN
		Z	<=	"00";
		CASE (PS) IS
			WHEN ST00 =>
				Z(1)	<=	'1';
				IF ( X = '0' ) THEN
					Z(2)	<=	'0';
					NS		<=	ST10;
				ELSIF ( X = '1' ) THEN
					Z(2)	<=	'0';
					NS		<=	ST00;
				END IF;
			WHEN ST01 =>
				Z(1)	<=	'0';
				IF ( X = '0' ) THEN
					Z(2)	<=	'0';
					NS		<=	ST11;
				ELSIF ( X = '1' ) THEN
					Z(2)	<=	'0';
					NS		<=	ST01;
				END IF;
			WHEN ST10 =>
				Z(1)	<=	'1';
				IF ( X = '0' ) THEN
					Z(2)	<=	'0';
					NS		<=	ST01;
				ELSIF ( X = '1' ) THEN
					Z(2)	<=	'0';
					NS		<=	ST00;
				END IF;
			WHEN ST11 =>
				Z(1)	<=	'0';
				IF ( X = '0' ) THEN
					Z(2)	<=	'1';
					NS		<=	ST00;
				ELSIF ( x = '1' ) THEN
					Z(2)	<=	'0';
					NS		<=	ST01;
				END IF;
			WHEN OTHERS =>
				Z	<=	"00";
				NS	<=	ST00;
		END CASE;
	END PROCESS comb_process;
	
	WITH PS SELECT
		Y	<=	"00" WHEN ST00,
				"01" WHEN ST01,
				"10" WHEN ST10,
				"11" WHEN ST11,
				"00" WHEN OTHERS;
END Exercise_8_6_arc;
	\end{lstlisting}
	
	\item The state machine is of Mealy type.
	
	\begin{minipage}[t]{0.64\textwidth}
		\vspace{0pt}\raggedright
		\centering
		\begin{tikzpicture}[>=stealth',shorten >=1pt,auto,node distance=2cm]
		
		%FSM states
		\node[state, initial, initial text=SET] (ST0) {A};
		\node[state, initial, initial text=CLR] (ST3) [below= of ST0] {D};
		\node[state] (ST1) [right= of ST0] {B};
		\node[state] (ST2) [right= of ST3] {C};
		
		% FSM state changes/variables
		\path[->]  (ST0)  edge [loop above]  node {X1/\NOTT{Z1}/\NOTT{Z2}} (ST0);
		\path[->]  (ST1)  edge [loop above]  node {\NOTT{X2}/Z1/\NOTT{Z2}} (ST1);
		\path[->]  (ST2)  edge [loop below]  node {\NOTT{X2}/\NOTT{Z1}/Z2} (ST2);
		\path[->]  (ST3)  edge [loop below]  node {X1/Z1/Z2} (ST3);
		\path[->]  (ST0) 	edge [bend left=18]  	node {\NOTT{X1}/\NOTT{Z1}/\NOTT{Z2}} (ST1);
		\path[->]  (ST1)    edge [bend left=18]     node {X2/Z1/Z2} (ST2);
		\path[->]  (ST2)    edge [bend left=18]     node {X2/\NOTT{Z1}/\NOTT{Z2}} (ST1);
		\path[->]  (ST3)    edge [bend right=18]    node[below] {\NOTT{X1}/Z1/Z2} (ST2);
		
		% legend
		\node[state]  (legend) [below left = 1 cm and 2 cm of ST0.east,font=\scriptsize] {State};
		\node[below =.1cm of legend, align=left, font=\scriptsize] (S) {input/Z1/Z2};
		\end{tikzpicture}
	\end{minipage}

	\item \begin{lstlisting}
LIBRARY IEEE;
USE IEEE.STD_LOGIC_1164.ALL;

ENTITY Exercise_8_8 IS
	PORT(
		CLK,X	:	IN	STD_LOGIC;
		Y		:	OUT	STD_LOGIC_VECTOR(2 DOWNTO 0);
		Z		:	OUT	STD_LOGIC_VECTOR(1 TO 2));
END Exercise_8_8;

ARCHITECTURE Exercise_8_8_arc OF Exercise_8_8 IS
	TYPE STATE_TYPE IS (ST000,ST001,ST010,ST011,ST100,ST101,ST110,ST111);
	SIGNAL PS,NS	:	STATE_TYPE;
BEGIN
	sync_process:	PROCESS(CLK) IS
	BEGIN
		IF(RISING_EDGE(CLK)) THEN
			PS	<=	NS;
		END IF;
	END PROCESS sync_process;
	
	comb_process:	PROCESS(X,PS) IS
	BEGIN
		Z	<=	"00";
		CASE (PS) IS
			WHEN ST000 =>
				Z	<=	"00";
				IF ( X = '0' ) THEN
					NS	<=	ST000;
				ELSIF ( X = '1' ) THEN
					NS	<=	ST001;
				END IF;
			WHEN ST001 =>
				Z	<=	"00";
				IF ( X = '0' ) THEN
					NS	<=	ST001;
				ELSIF ( X = '1' ) THEN
					NS	<=	ST010;
				END IF;
			WHEN ST010 =>
				Z	<=	"00";
				IF ( X = '0' ) THEN
					NS	<=	ST010;
				ELSIF ( X = '1' ) THEN
					NS	<=	ST011;
				END IF;
			WHEN ST011 =>
				Z	<=	"10";
				IF ( X = '0' ) THEN
					NS	<=	ST011;
				ELSIF ( X = '1' ) THEN
					NS	<= ST100;
				END IF;
			WHEN ST100 =>
				Z	<=	"01";
				If ( X = '0' ) THEN
					NS	<=	ST000;
				ELSIF ( X = '1' ) THEN
					NS	<=	ST101;
				END IF;
			WHEN ST101 =>
				Z	<=	"11";
				IF ( X = '0' ) THEN
					NS	<=	ST101;
				ELSIF ( X = '1' ) THEN
					NS	<=	ST110;
				END IF;
			WHEN ST110 =>
				Z	<=	"11";
				IF ( X = '0' ) THEN
					NS	<=	ST110;
				ELSIF ( X = '1' ) THEN
					NS	<=	ST111;
				END IF;
			WHEN ST111 =>
				Z	<=	"11";
				IF ( X = '0' ) THEN
					NS	<=	ST111;
				ELSIF ( X = '1' ) THEN
					NS	<=	ST000;
				END IF;
			WHEN OTHERS =>
				Z	<=	"00";
				NS	<=	ST000;
		END CASE;
	END PROCESS comb_process;
	
		WITH PS SELECT
		Y	<=	"000" WHEN ST000,
				"001" WHEN ST001,
				"010" WHEN ST010,
				"011" WHEN ST011,
				"100" WHEN ST100,
				"101" WHEN ST101,
				"110" WHEN ST110,
				"111" WHEN ST111,
				"000" WHEN OTHERS;
END Exercise_8_8_arc;
	\end{lstlisting}
	
	\item \begin{lstlisting}
LIBRARY IEEE;
USE IEEE.STD_LOGIC_1164.ALL;

ENTITY Exercise_8_9 IS
	PORT(
		CLK,X	:	IN	STD_LOGIC;
		Y		:	OUT	STD_LOGIC_VECTOR(1 DOWNTO 0);
		Z		:	OUT	STD_LOGIC_VECTOR(1 TO 2));
END Exercise_8_9;

ARCHITECTURE Exercise_8_9_arc OF Exercise_8_9 IS
	TYPE STATE_TYPE IS (ST00,ST01,ST10,ST11);
	SIGNAL PS,NS	:	STATE_TYPE;
BEGIN
	sync_process:	PROCESS(CLK) IS
	BEGIN
		IF(RISING_EDGE(CLK)) THEN
			PS	<=	NS;
		END IF;
	END PROCESS sync_process;
	
	comb_process:	PROCESS(X,PS) IS
	BEGIN
		Z	<=	"00";
		CASE (PS) IS
			WHEN ST00 =>
				Z(1)	<=	'0';
				IF ( X = '0' ) THEN
					Z(2)	<=	'0';
					NS		<=	ST00;
				ELSIF ( X = '1' ) THEN
					Z(2)	<=	'0';
					NS		<=	ST01;
				END IF;
			WHEN ST01 =>
				Z(1)	<=	'0';
				IF ( X = '0' ) THEN
					Z(2)	<=	'0';
					NS		<=	ST00;
				ELSIF ( X = '1' ) THEN
					Z(2)	<=	'0';
					NS		<=	ST10;
				END IF;
			WHEN ST10 =>
				Z(1)	<=	'0';
				IF ( X = '0' ) THEN
					Z(2)	<=	'0';
					NS		<=	ST00;
				ELSIF ( x = '1' ) THEN
					Z(2)	<=	'1';
					NS		<=	ST11;
				END IF;
			WHEN ST11 =>
				Z(1)	<=	'1';
				IF ( X = '0' ) THEN
					Z(2)	<=	'0';
					NS		<=	ST00;
				ELSIF ( X = '1' ) THEN
					Z(2)	<=	'1';
					NS		<=	ST11;
				END IF;
			WHEN OTHERS =>
				Z	<=	"00";
				NS	<=	ST00;
		END CASE;
	END PROCESS comb_process;
	
	WITH PS SELECT
		Y	<=	"00" WHEN ST00,
				"01" WHEN ST01,
				"10" WHEN ST10,
				"11" WHEN ST11,
				"00" WHEN OTHERS;
END Exercise_8_9_arc;
	\end{lstlisting}
	
	\item \begin{lstlisting}
LIBRARY IEEE;
USE IEEE.STD_LOGIC_1164.ALL;

ENTITY Exercise_8_10 IS
	PORT(
		X,CLK	:	IN	STD_LOGIC;
		Y		:	OUT	STD_LOGIC_VECTOR(1 TO 3);
		Z		:	OUT	STD_LOGIC);
END Exercise_8_10;

ARCHITECTURE Exercise_8_10_arc OF Exercise_8_10 IS
	TYPE STATE_TYPE IS (ST001,ST010,ST100);
	SIGNAL PS,NS	:	STATE_TYPE;
BEGIN
	sync_process:	PROCESS(CLK) IS
	BEGIN
		IF(RISING_EDGE(CLK)) THEN
			PS	<=	NS;
		END IF;
	END PROCESS sync_process;
	
	comb_process:	PROCESS(X,PS) IS
	BEGIN
		Z	<=	'0';
		CASE (PS) IS
			WHEN ST100 =>
				Z	<=	'1';
				NS	<=	ST010;
			WHEN ST010 =>
				IF ( X = '0' ) THEN
					Z	<=	'0';
					NS	<=	ST100;
				ELSIF ( X = '1' ) THEN
					Z	<=	'0';
					NS	<=	ST001;
				END IF;
			WHEN ST001 =>
				Z	<=	'1';
				NS	<=	ST100;
			WHEN OTHERS =>
				Z	<=	'0';
				NS	<=	ST100;
		END CASE;				
	END PROCESS comb_process;
	
	WITH PS SELECT
		Y	<=	"100" WHEN ST100,
				"010" WHEN ST010,
				"001" WHEN ST001,
				"100" WHEN OTHERS;
END Exercise_8_10_arc;
	\end{lstlisting}
	
	\item \begin{lstlisting}
LIBRARY IEEE;
USE IEEE.STD_LOGIC_1164.ALL;

ENTITY Exercise_8_10 IS
	PORT(
		X		:	IN	STD_LOGIC_VECTOR(1 TO 2);
		CLK		:	IN	STD_LOGIC;
		Y		:	OUT	STD_LOGIC_VECTOR(2 DOWNTO 0);
		Z		:	OUT	STD_LOGIC);
END Exercise_8_10;

ARCHITECTURE Exercise_8_10_arc OF Exercise_8_10 IS
	TYPE STATE_TYPE IS (ST001,ST010,ST100);
	SIGNAL PS,NS	:	STATE_TYPE;
BEGIN
	sync_process:	PROCESS(CLK) IS
	BEGIN
		IF(RISING_EDGE(CLK)) THEN
			PS	<=	NS;
		END IF;
	END PROCESS sync_process;
	
	comb_process:	PROCESS(X,PS) IS
	BEGIN
		Z	<=	'0';
		CASE (PS) IS
			WHEN ST001 =>
				IF ( X(1) = '0' ) THEN
					Z	<=	'1';
					NS	<=	ST010;
				ELSIF ( X(1) = '1' ) THEN
					Z	<=	'1';
					NS	<=	ST100;
				END IF;
			WHEN ST010 =>
				IF ( X(1) = '0' ) THEN
					Z	<=	'0';
					NS	<=	ST100;
				ELSIF ( X(1) = '1' ) THEN
					Z	<=	'0';
					NS	<=	ST010;
				END IF;
			WHEN ST100 =>
				IF ( X(2) = '0' ) THEN
					Z	<=	'0';
					NS	<=	ST001;
				ELSIF ( X(2) = '1' ) THEN
					Z	<=	'1';
					NS	<=	ST100;
				END IF;
			WHEN OTHERS =>
				Z	<=	'0';
				NS	<=	ST100;
		END CASE;				
	END PROCESS comb_process;
	
	WITH PS SELECT
		Y	<=	"100" WHEN ST100,
				"010" WHEN ST010,
				"001" WHEN ST001,
				"100" WHEN OTHERS;
END Exercise_8_10_arc;
	\end{lstlisting}
	
	\item \begin{lstlisting}
LIBRARY IEEE;
USE IEEE.STD_LOGIC_1164.ALL;

ENTITY Exercise_8_12 IS
	PORT(
		CLK	:	IN	STD_LOGIC;
		X	:	IN	STD_LOGIC_VECTOR(1 TO 2);
		Y	:	OUT	STD_LOGIC_VECTOR(2 DOWNTO 1);
		Z	:	OUT	STD_LOGIC_VECTOR(1 TO 2));
END Exercise_8_12;

ARCHITECTURE Exercise_8_12_arc OF Exercise_8_12 IS
	TYPE STATE_TYPE IS (ST00,ST01,ST11);
	SIGNAL PS,NS	:	STATE_TYPE;
BEGIN
	sync_process:	PROCESS(CLK) IS
	BEGIN
		IF(RISING_EDGE(CLK)) THEN
			PS	<=	NS;
		END IF;
	END PROCESS sync_process;
	
	comb_process:	PROCESS(X,PS) IS
	BEGIN
		Z	<=	"00";
		CASE (PS) IS
			WHEN ST00 =>
				Z(2)	<=	'1';
				IF ( X(2) = '0' ) THEN
					Z(1)	<=	'0';
					NS		<=	ST11;
				ELSIF ( X(2) = '1' ) THEN
					Z(1)	<=	'1';
					NS		<=	ST00;
				END IF;
			WHEN ST01 =>
				Z(2)	<=	'0';
				IF ( X(2) = '0' ) THEN
					Z(1)	<=	'1';
					NS		<=	ST00;
				ELSIF ( X(2) = '1' ) THEN
					Z(1)	<=	'0';
					NS		<=	ST11;
				END IF;
			WHEN ST11 =>
				Z(2)	<=	'1';
				IF ( X(1) = '0' ) THEN
					Z(1)	<=	'0';
					NS		<=	ST11;
				ELSIF ( X(1) = '1' ) THEN
					Z(1)	<=	'1';
					NS		<=	ST01;
				END IF;
			WHEN OTHERS =>
				Z	<=	"00";
				NS	<=	ST00;
		END CASE;
	END PROCESS comb_process;
	
	WITH PS SELECT
		Y	<=	"00" WHEN ST00,
				"01" WHEN ST01,
				"11" WHEN ST11,
				"00" WHEN OTHERS;
END Exercise_8_12_arc;
	\end{lstlisting}
	
	\item \begin{lstlisting}
LIBRARY IEEE;
USE IEEE.STD_LOGIC_1164.ALL;

ENTITY Exercise_8_13 IS
	PORT(
		X			:	IN	STD_LOGIC_VECTOR(1 TO 2);
		CLK			:	IN	STD_LOGIC;
		Y			:	OUT	STD_LOGIC_VECTOR(3 DOWNTO 1);
		CS,RD		:	OUT	STD_LOGIC);
END Exercise_8_13;

ARCHITECTURE Exercise_8_13_arc OF Exercise_8_13 IS
	TYPE STATE_TYPE IS (ST001,ST010,ST100);
	SIGNAL PS,NS	:	STATE_TYPE;
BEGIN
	sync_process:	PROCESS(CLK) IS
	BEGIN
		IF(RISING_EDGE(CLK)) THEN
			PS	<=	NS;
		END IF;
	END PROCESS sync_process;
	
	comb_process:	PROCESS(X,PS) IS
	BEGIN
		CS	<=	'0';
		RD	<=	'0';
		CASE (PS) IS
			WHEN ST001 =>
				IF ( X(1) = '0' ) THEN
					CS	<=	'0';
					RD	<=	'1';
					NS	<=	ST010;
				ELSIF ( X(1) = '1' ) THEN
					CS	<=	'1';
					RD	<=	'0';
					NS	<=	ST100;
				END IF;
			WHEN ST010 =>
				CS	<=	'1';
				RD	<=	'1';
				NS	<=	ST100;
			WHEN ST100 =>
				IF ( X(2) = '0' ) THEN
					CS	<=	'0';
					RD	<=	'0';
					NS	<=	ST001;
				ELSIF ( X(2) = '1' ) THEN
					CS	<=	'0';
					RD	<=	'1';
					NS	<=	ST100;
				END IF;
			WHEN OTHERS =>
				CS	<=	'0';
				RD	<=	'0';
				NS	<=	ST100;
		END CASE;				
	END PROCESS comb_process;
	
	WITH PS SELECT
		Y	<=	"100" WHEN ST100,
				"010" WHEN ST010,
				"001" WHEN ST001,
				"100" WHEN OTHERS;
END Exercise_8_13_arc;
	\end{lstlisting}
	
	\item \begin{lstlisting}
LIBRARY IEEE;
USE IEEE.STD_LOGIC_1164.ALL;

ENTITY Exercise_8_14 IS
	PORT(
		X		:	IN	STD_LOGIC_VECTOR(1 TO 2);
		CLK		:	IN	STD_LOGIC;
		Y		:	OUT	STD_LOGIC_VECTOR(3 DOWNTO 1);
		Z		:	OUT	STD_LOGIC_VECTOR(1 TO 2));
END Exercise_8_14;

ARCHITECTURE Exercise_8_14_arc OF Exercise_8_14 IS
	TYPE STATE_TYPE IS (ST001,ST010,ST100);
	SIGNAL PS,NS	:	STATE_TYPE;
BEGIN
	sync_process:	PROCESS(CLK) IS
	BEGIN
		IF(RISING_EDGE(CLK)) THEN
			PS	<=	NS;
		END IF;
	END PROCESS sync_process;
	
	comb_process:	PROCESS(X,PS) IS
	BEGIN
		Z	<=	"00";
		CASE (PS) IS
			WHEN ST001 =>
				Z(1)	<=	'0';
				IF ( X(1) = '0' ) THEN
					Z(2)	<=	'0';
					NS	<=	ST100;
				ELSIF ( X(1) = '1' ) THEN
					Z(2)	<=	'1';
					NS	<=	ST010;
				END IF;
			WHEN ST010 =>
				Z(1)	<=	'1';
				IF ( X(2) = '0' ) THEN
					Z(2)	<=	'1';
					NS	<=	ST100;
				ELSIF ( X(2) = '1' ) THEN
					Z(2)	<=	'0';
					NS	<=	ST001;
				END IF;
			WHEN ST100 =>
				Z(1)	<=	'1';
				IF ( X(1) = '0' ) THEN
					Z(2)	<=	'1';
					NS	<=	ST001;
				ELSIF ( X(1) = '1' ) THEN
					Z(2)	<=	'1';
					NS	<=	ST010;
				END IF;
			WHEN OTHERS =>
				Z	<=	"00";
				NS	<=	ST010;
		END CASE;				
	END PROCESS comb_process;
	
	WITH PS SELECT
		Y	<=	"100" WHEN ST100,
				"010" WHEN ST010,
				"001" WHEN ST001,
				"100" WHEN OTHERS;
END Exercise_8_14_arc;
	\end{lstlisting}
\end{enumerate}

\section*{Chapter 9}

\section*{Chapter 10}
The components used in each of the exercises are as shown below.
\begin{lstlisting}
----------------------
--- 8-bit, 2:1 Mux ---
----------------------

LIBRARY IEEE;
USE IEEE.STD_LOGIC_1164.ALL;

ENTITY mux2i1o8w IS
	PORT(
		in1,in2	:	IN	STD_LOGIC_VECTOR(7 DOWNTO 0);
		sel		:	IN	STD_LOGIC;
		out1	:	OUT	STD_LOGIC_VECTOR(7 DOWNTO 0));
END mux2i1o8w;

ARCHITECTURE mux2i1o8w OF mux2i1o8w IS
BEGIN
	WITH sel SELECT
		out1	<=	in1 WHEN '0',
					in2 WHEN '1',
					(OTHERS => '0') WHEN OTHERS;
END mux2i1o8w;

----------------------
--- 8-bit, 4:1 Mux ---
----------------------

LIBRARY IEEE;
USE IEEE.STD_LOGIC_1164.ALL;

ENTITY mux4i1o8w IS
	PORT(
		in0,in1,in2,in3	:	IN	STD_LOGIC_VECTOR(7 DOWNTO 0);
		sel				:	IN	STD_LOGIC_VECTOR(1 DOWNTO 0);
		out1			:	OUT	STD_LOGIC_VECTOR(7 DOWNTO 0));
END mux4i1o8w;

ARCHITECTURE mux4i1o8w OF mux4i1o8w IS
BEGIN
	WITH sel SELECT
		out1	<=	in0 WHEN "00",
					in1 WHEN "01",
					in2 WHEN "10",
					in3 WHEN "11",
					(OTHERS => '0') WHEN OTHERS;
END mux4i1o8w;



-------------------
--- 1:2 Decoder ---
-------------------

LIBRARY IEEE;
USE IEEE.STD_LOGIC_1164.ALL;

ENTITY decoder1to2 IS
	PORT(
		in1			:	IN	STD_LOGIC;
		out0,out1	:	OUT	STD_LOGIC);
END decoder1to2;

ARCHITECTURE dec1to2 OF decoder1to2 IS
BEGIN
	output_process:	PROCESS (in1) IS
	BEGIN
		IF (in1 = '0') THEN
			out0 <= '1';
			out1 <= '0';
		ELSIF (in1 = '1') THEN
			out0 <= '0';
			out1 <= '1';
		ELSE
			out0 <= '0';
			out1 <= '0';
		END IF;
	END PROCESS output_process;
END dec1to2;

----------------------
--- 8-bit Register ---
----------------------

LIBRARY IEEE;
USE IEEE.STD_LOGIC_1164.ALL;

ENTITY reg8 IS
	PORT(
		REG_IN	:	IN	STD_LOGIC_VECTOR(7 DOWNTO 0);
		CLK,LD	:	IN	STD_LOGIC;
		REG_OUT	:	OUT	STD_LOGIC_VECTOR(7 DOWNTO 0));
END reg8;

ARCHITECTURE reg OF reg8 IS
BEGIN
	load_process: PROCESS(CLK) IS
	BEGIN
		IF (RISING_EDGE(CLK)) THEN
			IF (LD = '1') THEN
				REG_OUT <= REG_IN;
			END IF;
		END IF;
	END PROCESS load_process;
END reg;
\end{lstlisting}
\begin{enumerate}
	\item \begin{lstlisting}
LIBRARY IEEE;
USE IEEE.STD_LOGIC_1164.ALL;

ENTITY muxreg IS
	PORT(
		A,B			:	IN	STD_LOGIC_VECTOR(7 DOWNTO 0);
		LDA,SEL,CLK	:	IN	STD_LOGIC;
		F			:	OUT	STD_LOGIC_VECTOR(7 DOWNTO 0));
END muxreg;

ARCHITECTURE muxreg_arc OF muxreg IS
	
	----------------------
	--- 8-bit Register ---
	----------------------
	
	COMPONENT reg8 IS
		PORT(
			REG_IN	:	IN	STD_LOGIC_VECTOR(7 DOWNTO 0);
			CLK,LD	:	IN	STD_LOGIC;
			REG_OUT	:	OUT	STD_LOGIC_VECTOR(7 DOWNTO 0));
	END COMPONENT reg8;
	
	----------------------
	--- 8-bit, 2:1 Mux ---
	----------------------
	
	COMPONENT mux2i1o8w IS
		PORT(
			in1,in2	:	IN	STD_LOGIC_VECTOR(7 DOWNTO 0);
			sel		:	IN	STD_LOGIC;
			out1	:	OUT	STD_LOGIC_VECTOR(7 DOWNTO 0));
	END COMPONENT mux2i1o8w;
	
	SIGNAL muxout	:	STD_LOGIC_VECTOR(7 DOWNTO 0);
	
BEGIN
	
	mux:	mux2i1o8w PORT MAP (
							in1		=> A,
							in2		=> B,
							sel		=> SEL,
							out1	=> muxout);
								
	reg:	reg8 PORT MAP (
							REG_IN	=>	muxout,
							CLK		=>	CLK,
							LD		=>	LDA,
							REG_OUT	=>	F);
	
END muxreg_arc;
	\end{lstlisting}
	
	\item \begin{lstlisting}
LIBRARY IEEE;
USE IEEE.STD_LOGIC_1164.ALL;

ENTITY muxdec2r IS
	PORT(
		X,Y,Z	:	IN	STD_LOGIC_VECTOR(7 DOWNTO 0);
		MS		:	IN	STD_LOGIC_VECTOR(1 DOWNTO 0);
		DS,CLK	:	IN	STD_LOGIC;
		RB,RA	:	OUT	STD_LOGIC_VECTOR(7 DOWNTO 0));
END muxdec2r;

ARCHITECTURE muxdec2r OF muxdec2r IS
	
	----------------------
	--- 8-bit, 4:1 Mux ---
	----------------------
	
	COMPONENT mux4i1o8w IS
		PORT(
		in0,in1,in2,in3	:	IN	STD_LOGIC_VECTOR(7 DOWNTO 0);
			sel				:	IN	STD_LOGIC_VECTOR(1 DOWNTO 0);
			out1			:	OUT	STD_LOGIC_VECTOR(7 DOWNTO 0));
	END COMPONENT mux4i1o8w;
	
	
	-------------------
	--- 1:2 Decoder ---
	-------------------
	
	COMPONENT decoder1to2 IS
		PORT(
			in1			:	IN	STD_LOGIC;
			out0,out1	:	OUT	STD_LOGIC);
	END COMPONENT decoder1to2;

	
	----------------------
	--- 8-bit Register ---
	----------------------

	COMPONENT reg8 IS
		PORT(
			REG_IN	:	IN	STD_LOGIC_VECTOR(7 DOWNTO 0);
			CLK,LD	:	IN	STD_LOGIC;
			REG_OUT	:	OUT	STD_LOGIC_VECTOR(7 DOWNTO 0));
	END COMPONENT reg8;
	
	SIGNAL muxout,regAout,regBout	:	STD_LOGIC_VECTOR(7 DOWNTO 0);
	SIGNAL decout0,decout1			:	STD_LOGIC;

BEGIN
	
	mux:	mux4i1o8w PORT MAP (
		in0		=>	regBout,
		in1		=>	Z,
		in2		=>	Y,
		in3		=>	X,
		sel		=>	MS,
		out1	=> muxout);
		
	dec:	decoder1to2 PORT MAP (
		in1		=>	DS,
		out0	=>	decout0,
		out1	=>	decout1);
		
	regA:	reg8 PORT MAP (
		REG_IN	=>	muxout,
		CLK		=>	CLK,
		LD		=>	decout0,
		REG_OUT	=>	regAout);
		
	regB:	reg8 PORT MAP (
		REG_IN	=>	regAout,
		CLK		=>	CLK,
		LD		=>	decout1,
		REG_OUT	=>	regBout);
	
	RA <= regAout;
	RB <= regBout;
	
END muxdec2r;
	\end{lstlisting}
	
	\item \begin{lstlisting}
LIBRARY IEEE;
USE IEEE.STD_LOGIC_1164.ALL;

ENTITY DualMuxDualReg IS
	PORT(
		X,Y		:	IN	STD_LOGIC_VECTOR(7 DOWNTO 0);
		LDA,LDB	:	IN	STD_LOGIC;
		S0,S1	:	IN	STD_LOGIC;
		CLK		:	IN	STD_LOGIC;
		RB		:	OUT	STD_LOGIC_VECTOR(7 DOWNTO 0));
END DualMuxDualReg;

ARCHITECTURE DualMuxDualReg_Arc OF DualMuxDualReg IS
	
	----------------------
	--- 8-bit Register ---
	----------------------

	COMPONENT reg8 IS
		PORT(
			REG_IN	:	IN	STD_LOGIC_VECTOR(7 DOWNTO 0);
			CLK,LD	:	IN	STD_LOGIC;
			REG_OUT	:	OUT	STD_LOGIC_VECTOR(7 DOWNTO 0));
	END COMPONENT reg8;
	
	-------------------------------
	--- 2-in, 1-out, 8-wide Mux ---
	-------------------------------
	
	COMPONENT mux2i1o8w IS
		PORT(
			in1,in2	:	IN	STD_LOGIC_VECTOR(7 DOWNTO 0);
			sel		:	IN	STD_LOGIC;
			out1	:	OUT	STD_LOGIC_VECTOR(7 DOWNTO 0));
	END COMPONENT mux2i1o8w;
	
	------------------------
	--- Internal Signals ---
	------------------------
		
	SIGNAL Mux1Out,Mux2Out	:	STD_LOGIC_VECTOR(7 DOWNTO 0);
	SIGNAL RegAOut,RegBOut	:	STD_LOGIC_VECTOR(7 DOWNTO 0);
	
BEGIN
	
	mux1:	mux2i1o8w PORT MAP(
		in1		=>	X,
		in2		=>	RegBOut,
		sel		=>	S1,
		out1	=>	Mux1Out);
		
	regA:	reg8 PORT MAP(
		REG_IN	=>	Mux1Out,
		CLK		=>	CLK,
		LD		=>	LDA,
		REG_OUT	=>	RegAOut);
		
	mux2:	mux2i1o8w PORT MAP(
		in1		=>	RegAOut,
		in2		=>	Y,
		sel		=>	S0,
		out1	=>	Mux2Out);
		
	regB:	reg8 PORT MAP(
		REG_IN	=>	Mux2Out,
		CLK		=>	CLK,
		LD		=>	LDB,
		REG_OUT	=>	RegBOut);
		
	RB	<=	RegBOut;
	
END DualMuxDualReg_Arc;
	\end{lstlisting}
	
	\item \begin{lstlisting}
LIBRARY IEEE;
USE IEEE.STD_LOGIC_1164.ALL;

ENTITY DualMuxDualReg IS
	PORT(
		LDA,LDB,S0,S1,RD,CLK	:	IN	STD_LOGIC;
		X,Y						:	IN	STD_LOGIC_VECTOR(7 DOWNTO 0);
		RA,RB					:	OUT	STD_LOGIC_VECTOR(7 DOWNTO 0));
END DualMuxDualReg;

ARCHITECTURE DualMuxDualReg_arc OF DualMuxDualReg IS
	
	----------------------
	--- 8-bit Register ---
	----------------------

	COMPONENT reg8 IS
		PORT(
			REG_IN	:	IN	STD_LOGIC_VECTOR(7 DOWNTO 0);
			CLK,LD	:	IN	STD_LOGIC;
			REG_OUT	:	OUT	STD_LOGIC_VECTOR(7 DOWNTO 0));
	END COMPONENT reg8;
	
	-------------------------------
	--- 2-in, 1-out, 8-wide Mux ---
	-------------------------------
	
	COMPONENT mux2i1o8w IS
		PORT(
			in1,in2	:	IN	STD_LOGIC_VECTOR(7 DOWNTO 0);
			sel		:	IN	STD_LOGIC;
			out1	:	OUT	STD_LOGIC_VECTOR(7 DOWNTO 0));
	END COMPONENT mux2i1o8w;
	
	------------------------
	--- Internal Signals ---
	------------------------
		
	SIGNAL Mux1Out,Mux2Out	:	STD_LOGIC_VECTOR(7 DOWNTO 0);
	SIGNAL RegAOut,RegBOut	:	STD_LOGIC_VECTOR(7 DOWNTO 0);
	SIGNAL And1Out,And2Out	:	STD_LOGIC;
	
BEGIN
	
	And1Out	<=	LDB AND NOT RD;
	AND2Out	<=	LDA AND RD;
	
	mux1:	mux2i1o8w PORT MAP(
		in1		=>	X,
		in2		=>	Y,
		sel		=>	S1,
		out1	=>	Mux1Out);
		
	mux2:	mux2i1o8w PORT MAP(
		in1		=>	X,
		in2		=>	RegBOut,
		sel		=>	S0,
		out1	=>	Mux2Out);
		
	regA:	reg8 PORT MAP(
		REG_IN	=>	Mux2Out,
		CLK		=>	CLK,
		LD		=>	And2Out,
		REG_OUT	=>	RegAOut);
		
	regB:	reg8 PORT MAP(
		REG_IN	=>	Mux1Out,
		CLK		=>	CLK,
		LD		=>	And1Out,
		REG_OUT	=>	RegBOut);
		
	RA	<=	RegAOut;
	RB	<=	RegBOut;
	
END DualMuxDualReg_arc;
	\end{lstlisting}
	
	\item \begin{lstlisting}
LIBRARY IEEE;
USE IEEE.STD_LOGIC_1164.ALL;

ENTITY DualRegDecMux IS
	PORT(
		A,B,C		:	IN	STD_LOGIC_VECTOR(7 DOWNTO 0);
		SL1,SL2,CLK	:	IN	STD_LOGIC;
		RAX,RBX		:	OUT	STD_LOGIC_VECTOR(7 DOWNTO 0));
END DualRegDecMux;

ARCHITECTURE DualRegDecMux_arc OF DualRegDecMux IS
	
	-------------------
	--- 1:2 Decoder ---
	-------------------
	
	COMPONENT decoder1to2 IS
		PORT(
			in1			:	IN	STD_LOGIC;
			out0,out1	:	OUT	STD_LOGIC);
	END COMPONENT decoder1to2;
	
	----------------------
	--- 8-bit Register ---
	----------------------

	COMPONENT reg8 IS
		PORT(
			REG_IN	:	IN	STD_LOGIC_VECTOR(7 DOWNTO 0);
			CLK,LD	:	IN	STD_LOGIC;
			REG_OUT	:	OUT	STD_LOGIC_VECTOR(7 DOWNTO 0));
	END COMPONENT reg8;
	
	-------------------------------
	--- 2-in, 1-out, 8-wide Mux ---
	-------------------------------
	
	COMPONENT mux2i1o8w IS
		PORT(
			in1,in2	:	IN	STD_LOGIC_VECTOR(7 DOWNTO 0);
			sel		:	IN	STD_LOGIC;
			out1	:	OUT	STD_LOGIC_VECTOR(7 DOWNTO 0));
	END COMPONENT mux2i1o8w;
	
	------------------------
	--- Internal Signals ---
	------------------------
	
	SIGNAL DecOut0,DecOut1	:	STD_LOGIC;
	SIGNAL MuxOut			:	STD_LOGIC_VECTOR(7 DOWNTO 0);
	
BEGIN
	
	dec:	decoder1to2 PORT MAP (
		in1		=>	SL1,
		out0	=>	decout0,
		out1	=>	decout1);
		
	mux:	mux2i1o8w PORT MAP(
		in1		=>	B,
		in2		=>	C,
		sel		=>	SL2,
		out1	=>	MuxOut);
		
	regA:	reg8 PORT MAP(
		REG_IN	=>	A,
		CLK		=>	CLK,
		LD		=>	DecOut1,
		REG_OUT	=>	RAX);
		
	regB:	reg8 PORT MAP(
		REG_IN	=>	MuxOut,
		CLK		=>	CLK,
		LD		=>	DecOut0,
		REG_OUT	=>	RBX);
	
END DualRegDecMux_arc;
	\end{lstlisting}
	
	\item \begin{lstlisting}
LIBRARY IEEE;
USE IEEE.STD_LOGIC_1164.ALL;

ENTITY DualRegMuxDec IS
	PORT(
		A,B,C			:	IN	STD_LOGIC_VECTOR(7 DOWNTO 0);
		SEL1,SEL2,CLK	:	IN	STD_LOGIC;
		RAP,RBP			:	OUT	STD_LOGIC_VECTOR(7 DOWNTO 0));
END DualRegMuxDec;

ARCHITECTURE DualRegMuxDec_arc OF DualRegMuxDec IS
	
	-------------------
	--- 1:2 Decoder ---
	-------------------
	
	COMPONENT decoder1to2 IS
		PORT(
			in1			:	IN	STD_LOGIC;
			out0,out1	:	OUT	STD_LOGIC);
	END COMPONENT decoder1to2;
	
	----------------------
	--- 8-bit Register ---
	----------------------

	COMPONENT reg8 IS
		PORT(
			REG_IN	:	IN	STD_LOGIC_VECTOR(7 DOWNTO 0);
			CLK,LD	:	IN	STD_LOGIC;
			REG_OUT	:	OUT	STD_LOGIC_VECTOR(7 DOWNTO 0));
	END COMPONENT reg8;
	
	-------------------------------
	--- 2-in, 1-out, 8-wide Mux ---
	-------------------------------
	
	COMPONENT mux2i1o8w IS
		PORT(
			in1,in2	:	IN	STD_LOGIC_VECTOR(7 DOWNTO 0);
			sel		:	IN	STD_LOGIC;
			out1	:	OUT	STD_LOGIC_VECTOR(7 DOWNTO 0));
	END COMPONENT mux2i1o8w;
	
	------------------------
	--- Internal Signals ---
	------------------------
	
	SIGNAL DecOut0,DecOut1	:	STD_LOGIC;
	SIGNAL MuxOut			:	STD_LOGIC_VECTOR(7 DOWNTO 0);
	
BEGIN
	
	mux:	mux2i1o8w PORT MAP(
		in1		=>	A,
		in2		=>	B,
		sel		=>	SEL1,
		out1	=>	MuxOut);
		
	dec:	decoder1to2 PORT MAP (
		in1		=>	SEL2,
		out0	=>	decout0,
		out1	=>	decout1);
		
	regA:	reg8 PORT MAP(
		REG_IN	=>	MuxOut,
		CLK		=>	CLK,
		LD		=>	DecOut1,
		REG_OUT	=>	RAP);
		
	regB:	reg8 PORT MAP(
		REG_IN	=>	C,
		CLK		=>	CLK,
		LD		=>	DecOut0,
		REG_OUT	=>	RBP);
	
END DualRegMuxDec_arc;
	\end{lstlisting}
\end{enumerate}
\resumetocwriting